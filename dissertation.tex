%-----------------------------------------------------------------------
% Latex Thesis/Dissertation Template for Wright State University
%
% Written by Sean A. Mortara
% 28 June 2001
% Modified by Josh Mark
% 15 Dec 2011
% Later edits by Joseph C. Slater
%-----------------------------------------------------------------------
\documentclass[12pt]{report}

\usepackage{xcolor}
\usepackage{booktabs} % used for tables
\usepackage{multirow} % used for tables to merge multiple rows
\usepackage{bigdelim} % used for tables to set spacing
\usepackage{bigstrut} % used for tables to set spacing
\usepackage{graphicx} % used for includegraphics
%\usepackage{subfigure} % allows the use of subfigures
\usepackage{subcaption} % also allows the use of subfigures but not deprecated
%\usepackage[framed,numbered,autolinebreaks,useliterate]{mcode} % used to insert code
\usepackage{listings}
\usepackage{pgfgantt}
\usepackage{lscape}


%
%
%



%-----------------------------------------------------------------------
%  Modified fields
%-----------------------------------------------------------------------
\newcommand{\authorfirst}{Bradley}
\newcommand{\authorMI}{A.~}
\newcommand{\authorlast}{Schneider}
\newcommand{\degreefull}{Doc\-tor of Phil\-os\-o\-phy}  % force hyphenation at syllables if line breaks are required
\newcommand{\degreeshort}{Ph.D.}
\newcommand{\thesisordissertationlc}{Dissertation} % Make this uppercase
\newcommand{\dept}{Department of Computer Science and Engineering}
\newcommand{\institution}{Wright State University} % Doubting you will
                                % change this.
\newcommand{\thesistitle}{Bridging the Gap between Atomic and Complex
Activities in First Person Video using Fuzzy Inference} % Needes a line break
\newcommand{\bachdegreeshort}{B.S.} % Bachelor degree short
\newcommand{\bachinstitution}{Morehead State University} % Bachelor degree institution
\newcommand{\bachyear}{2012}% Bachelor degree year
\newcommand{\masterdegreeshort}{M.S.} % Bachelor degree short
\newcommand{\masterinstitution}{Wright State University} % Bachelor degree institution
\newcommand{\masteryear}{2017}% Bachelor degree year
%No spaces should be before or after this title.
\newcommand{\pdfsubject}{a short paraphrase of your title or focus of your thesis}
\newcommand{\pdfkeywords}{keyword 1, keyword 2, keyword 3, keyword 4}
\newcommand{\yearcomplete}{2020}
% set pdf file info
\usepackage{hyperxmp} % used to set pdf property info with \hypersetup command

%-----------------------------------------------------------------------
%  Thesis Advisor, Department Chair, Dean of Graduate Studies
%  I don't know why titles as separated... except in the one case at
%  the end.
%-----------------------------------------------------------------------
\newcommand{\thesisdirector}{Tanvi Banerjee}
\newcommand{\thesisdirectortitle}{Ph.D.}
\newcommand{\phdProgrameOrDeptChair}{Yong Pei} % CHANGE TO: Faculty M. Name FOR MASTER THESIS
\newcommand{\phdProgrameOrDeptChairEducation}{Ph.D.} % CHANGE TO: Ph.D., P.E. ETC
\newcommand{\phdProgrameOrDeptChairTitle}{Director, Computer Science and Engineering Ph.D. Program} % CHANGE TO: Chair, Department of Mechanical and \\ Materials Engineering FOR MASTER THESIS in MME


% COMMENT THE FOLLOWING 3 LINES FOR MASTER THESIS. PLEASE NOTE THAT FOR MASTER THESIS YOU SHOULD USE "\masterSignaturePage" AND FOR PHD DISSERATION "\phdSignaturePage" IN THE "approval sheet" SECTION.
\newcommand{\graduateSchoolDean}{Barry Milligan}
\newcommand{\graduateSchoolDeanEducation}{Ph.D.}
\newcommand{\graduateSchoolDeanTitle}{Interim Dean of the Graduate School}

%-----------------------------------------------------------------------
%  Final Examination Committee: Comment out the ones you don't need.
%-----------------------------------------------------------------------
\newcommand{\fecone}{Tanvi Banerjee, Ph.D.}

\newcommand{\fectwo}{Yong Pei, Ph.D.}

\newcommand{\fecthree}{Michael Riley, Ph.D.}

\newcommand{\fecfour}{Mateen Rizki, Ph.D.}

\newcommand{\fecfive}{Thomas Wischgoll, Ph.D.}

% If you have more committee members... good luck

% Modify this if needed for getting citations to "look right" according to your field. Read the natbib documentation on how to use this.
%\usepackage[round]{natbib}
%\usepackage{doublespace}

%=============================
%  Begin document!
%=============================
%
% Don't touch

% still don't touch.

% title sheet
\usepackage{WSUThesisTemplate/WSU}
\hypersetup{
                     pdfauthor={\authorfull},
                     pdftitle={\thesistitle},
                     pdfsubject={\pdfsubject},
                     pdfkeywords={\pdfkeywords},
                     }
% \normalem
\pagenumbering{roman}
\pagestyle{plain}
\rhead{\today}
\begin{document}
\maketitle
\doublespace\

% Still don't touch!!

%=============================
%  approval sheet
%=============================

\thispagestyle{empty}
\renewcommand\baselinestretch{2}
\begin{singlespace}
%\masterSignaturePage\newpage     % USE FOR MASTER THESIS
% You don't need both signature pages. Please comment out unnecessary lines in the .tex file. \newpage
\phdSignaturePage\newpage        % USE FOR PHD DISSERTATION
\end{singlespace}
%
%=============================
%  Abstract
%=============================
\newpage
\setcounter{page}{3}
\vspace{2in}
%
\begin{singlespace}
\begin{center}
  ABSTRACT
\end{center}
%
\noindent{\small{\authorlast, \authorfirst}.
		 {\degreeshort, \dept, \institution},
		 {\yearcomplete}.
		 {\sl \thesistitle}.}
\end{singlespace}
\vspace*{.5in}

\pdfbookmark[0]{Abstract}{Abstract}
%\phantomsection
%
%========================
% Start editing below.
%========================
Activities of Daily Living (ADL's) are the activities that people perform every day in their home as part of their typical routine. The in-home, automated monitoring of ADL's has broad utility for intelligent systems that enable independent living for the elderly and mentally or physically disabled individuals. With rising interest in electronic health (e-Health) and mobile health (m-Health) technology, opportunities abound for the integration of activity monitoring systems into these newer forms of healthcare.

In this dissertation we propose a novel system for describing ADL's based on video collected from a wearable camera. Most in-home activities are naturally defined by interaction with objects. We leverage these object-centric activity definitions to develop a set of rules for a Fuzzy Inference System (FIS) that uses video features and the identification of objects to identify and classify activities. Further, we demonstrate that the use of FIS enhances the reliability of the system and provides enhance explainability and interpretability of results over popular machine-learning classifiers due to the linguistic nature of fuzzy systems.



%%%%%%%%%%%%%%%%%%
%-----------------------------------------------------------------------
%
%=============================
%  Table of contents, etc.
%=============================
%\renewcommand\baselinestretch{1.5}
\begin{singlespace}
\tableofcontents
\listoffigures
\listoftables
\end{singlespace}
%
%=============================
%  Acknowledgments
%=============================
% \newpage
% \thispagestyle{plain}
% \setlength{\parindent}{0em}
% \begin{center}
% {\huge Acknowledgment}
% \end{center}

% I would like to take this opportunity to extend my thanks to\ldots\  If you have multiple paragraphs, the first should not be indented to match the style of the rest of the thesis.

% \setlength{\parindent}{2em}
% Any additional paragraphs should be indented as such.  That ugly command before this, you don't need to keep doing that. Remember to thank your advisor and committee members. Probably your Mom, others as you wish. 
%
%=============================
%  Dedication
%=============================
% \newpage
% \thispagestyle{plain}
% \vspace*{3in}
% \begin{center}
% Dedicated to\\
% Somebody special (Wife, husband, girlfriend or boyfriend works well
% here.)
% \end{center}
%
%
%
%=============================
%  Begin Chapters
%=============================
\newpage
\setcounter{page}{1}
\pagenumbering{arabic}
\setlength{\parindent}{2em}

%================================================
\chapter{Introduction}
%================================================

%-------------------------
\section{Need for In-Home Monitoring}
%-------------------------
The number of people aged 65 and older in the United States is expected to reach 89 million by 2050, double the number of United States citizens in the same age group in 2011 \cite{Jacobsen2011AmericasPopulation}. As the elderly population continues to grow, so will the need to provide in-home care for the elderly. Elder care has traditionally been lacking qualified, certified practitioners and clinicians, and the demand for in-home health aides is very likely to exceed recruitment rates over this period of growth \cite{Rowe2016PreparingPopulation}. This will result in a growing deficit of qualified healthcare workers for in-home care, which will require the development of new models of healthcare delivery.

One response to this deficit of personnel is to leverage e-Health technology to provide in-home healthcare. In a 2005 report, the World Health Organization (WHO) defined e-Health as the “cost-effective and secure use of information and communications technologies in support of health and health-related fields, including health-care services, health surveillance, health literature, and health education, knowledge and research” and resolved to encourage long-term strategic plans to develop e-Health resources, noting that advances in technology have raised expectations for healthcare \cite{58thWorldHealthAssembly2005WHA58.28EHealth}. Since then, WHO has also formalized the idea of mobile-health, or m-Health, which is similar to e-Health but uses mobile technology. This includes in-home installation of sensors or imaging devices used to monitor patients with chronic illnesses \cite{WorldHealthOrganization2011MHealth:EHealth}. A recent survey of m-Health technology found that m-Health has been found as an effective way to monitor elderly patients suffering from dementia and cognitive disorders including screening for cognitive decline and promoting healthy habits in terms of physical activity \cite{Vazquez2018E-HealthReview}.

The need for in-home monitoring extends beyond caring for the elderly to monitoring individuals of any age who require assistance living independently due to a mental or physical disability or injury. In addition to the use of in-home monitoring for clinical purposes, it may be an important aspect of ubiquitous smart home systems. As more and more electronic devices are becoming internet-of-things (IoT)-enabled, including televisions, kitchen appliances, and security systems, the demand is growing for systems that automatically understand and react to users and their actions.


%-------------------------
\section{Activities of Daily Living}
%-------------------------
Activities of Daily Living (ADL’s) are defined as activities routinely performed in daily life, often for basic hygiene and personal care as well as food preparation, housekeeping, and other aspects of independent living. These activities have been studied since at least the 1960s \cite{LawtonAssessmentLiving} as an assessment for the ability to perform physical self-maintenance, especially among the elderly. Several scales of assessment have been proposed, each with the common purpose of evaluating the cognitive and physical capabilities of the subject \cite{LawtonAssessmentLiving, Bucks1996AssessmentScale, Carswell1993ActivitiesDisease}. In the clinical setting, it has been shown that the ability to independently perform ADLs is an indicator of the onset and progression of conditions such as Dementia and Alzheimer's disease \cite{Desai2004ActivitiesTreatment}.

The manual assessment of human activities is dependent on the ability for the subject to be observed, necessitating the use of either in-home visits by a clinician or visits by the subject to clinical laboratories. This can be challenging when considering the expense of home visits and the limited ability of elderly to travel and coordinate appointments. The use of technology through eHealth and mHealth applications to monitor and observe ADLs in a home environment can greatly reduce the impact to both the subject and clinician by providing an unobtrusive, automated method of gathering activity data. [what to cite?]

A challenge to implementing automated activity detection and recognition systems is the need for a formalized model defining each activity under observation. The ability to recognize activities as they occur comes very naturally for humans, but it is much more difficult to define for a machine. Human activities are defined in overlapping temporal and semantic spaces. That is, one higher-level activity may include the performance of several lower-level activities at the same time. Thus the definition of these activities may be subjective to the observer. 

For example, a subject may perform the high-level activity of preparing breakfast, which involves many actions and movements in a sequence over a significant period of time. Within the action of preparing breakfast, the subject may have accomplished the lower-level activity of making toast, and within that activity the subject may have accomplished the lower-level activity of inserting bread into the toaster. The chain of activities may continue decomposing all the way down to the activity of picking up a bag of bread. Understanding the same or overlapping activities at all of these levels is crucial to a fully functioning activity monitoring system.


%-------------------------
\section{Research Goals}
%-------------------------
\emph{Research Question 1: Can wearable sensors be used to unobtrusively monitor activities of subjects in a semi-structured environment?}
A desired outcome of this work is to provide evidence that a system based on wearable sensors may be used in the home (i.e. not in a clinical setting or laboratory) to provide continuous monitoring of activities of daily living. The outputs of the system provide data on activities and patterns of activities to inform clinicians on possible changes in the health and wellness of the user. Data collected at clinical laboratories is temporally sparse and involves scripted movements, bringing into question the validity of the data for assessing the subject’s ability to perform the actions in a natural environment; by contrast, the data collected by in-home monitoring will show a more complete look into the subjects’ abilities to perform activities on a day-to-day basis.
Early work was focused on detecting and measuring gait in controlled and semi-controlled environments \cite{Schneider2017PreliminaryProcessing, Schneider2019ComparisonEnvironments}. While results were achieved with uninterrupted gait sequences of several meters, it was clear that long uninterrupted gait sequences are uncommon in many daily activities, which include starts and stops and navigation around household objects. For this reason, the ultimate focus of this work is on object interactions rather than gait, as object interactions provide a much richer context for activities that are being performed.

\emph{Research Question 2: Can features extracted from first-person videos recorded in subjects’ homes be used to categorize different interactions with objects?}
The use of object interactions to determine activities has been popular in recent automated activity recognition and classification systems \cite{Pirsiavash2012, Sudhakaran2018, Nakatani2018PreliminaryKnowledge, Wang2018, Gokce2019HumanPairs}. Many in-home ADL’s are at least partially defined by interactions with certain objects. For example, knowing that a subject is interacting with a remote control should reveal that the subject is interacting with a television or other electronic device. The search space of candidate activities can be greatly culled by knowing which objects are involved. A desired outcome of this work is to exploit a combination of video and object features to characterize different interactions with objects.
Building on an initial set of features from previous work with gait, an initial baseline set of features was identified for categorizing object interactions. In addition to the location of objects, this set of features was heavily dependent on optical flow features to identify changes in object location and color features to identify changes in the appearance of objects. Additional experiments are needed to extend and refine this set of features and the resulting Fuzzy Inference System (FIS) that was developed, but the initial work proved the feasibility of using first-person video features to categorize object interactions.

\emph{Research Question 3: Can the activity identification system a) accurately describe activities and b) handle uncertainty in the results better than current state-of-the-art methods?}
While much recent work has been done on using deep learning and neural networks to identify human activities in video sequences (such as run, walk, sit down, throw, turn on faucet, open refrigerator, etc.) \cite{Abebe2016, Ozkan2017, Li2016}, there are two major drawbacks to the current deep learning trend:
\begin{enumerate}
    \item The networks must be trained to recognize a very specific set of actions. To date, most work involving neural networks has been against small sets of precisely defined and scripted activities, limiting practical applications.
    \item Since deep learning requires no features to be identified/extracted, the use of these technologies in isolation provides no ability to explain the outcome – only to identify it. That is, when the network fails to provide the correct answer, there is no explainability of the result or the misidentified activity.
\end{enumerate}

We propose the use of fuzzy logic to address these limitations. Many activities are very similar in their basic movements and absolute identification may become difficult as the set of activities grows. If an unknown activity occurs, fuzzy outputs may still provide some insight into the activity or similar activities. This is an advantage over machine learning approaches which will simply fail to identify the behavior. The simplicity of a fuzzy system can be seen in different lights; our goal is to show that the simpler fuzzy model is capable of providing a rich description of activities. In the course of attempting to identify an activity, the fuzzy system will produce interpretable fuzzy variables (as opposed to unexplainable deep learning features) which describe actions in the scene  - for example, a specific object in the scene is being interacted with and the subject is manipulating the shape of the object. While the overall activity might be unidentified, a good amount of information may still be taken away from the fuzzy system.


%-------------------------
\section{Thesis Statement}
%-------------------------
This dissertation proposes to bridge the gap between in-home activity monitoring leveraging an object recognition framework and using a fuzzy linguistic framework to handle uncertainty in complex movements between humans and objects in an in-home unstructured setting.


%================================================
\chapter{Related Work}
%================================================
This chapter contains a summary of the existing body of work related to the objective of describing activities occurring in first person video. The related work varies based on several factors - the types of activities detected, the sensors used to detect the activities, and the computational intelligence methods used to convert sensor data into activity information. The chapter is organized into sections based on these factors. Each section provides an overview of the variation seen within the related work with respect to the given factor.

\section{Activities Being Detected by Vision Sensors}

A number of activities have been found in related work on detecting and classifying activities using vision sensors. These related works were grouped into three categories, and discussion of the types of activities follows below:
\begin{enumerate}
\item Anomaly Detection, 
\item Full-body Activities, and
\item Partial-body Activities.
\end{enumerate}

\subsection{Anomaly Detection}

A subset of related work focuses on the detection of anomalous activities. Each of these systems used full-body views of subjects from third-person vision sensors combined with computational intelligence to observe an environment and determine when either an unexpected activity occurred, or when an activity occurred in an unexpected way. Systems detecting such anomalies focus on actions that do not fit expectations based on previous observations, but do not often focus on describing precisely what these anomalous actions are - just that they are unexpected.

In \cite{Beleznai2012}, a vision sensor monitored a pedestrian area and used motion history images to identify activities that were determined to be anomalies, such as vehicles entering the space, by means of clustering. This method approximated the spatio-temporal distribution of activities that occurred frequently and detected activities that did not fit that distribution. Another outdoor environment - a loading dock - was monitored for anomalous activities in \cite{Hamid2005}. Activities were represented as n-grams of events and a distance metric based on these n-grams determined the dissimilarity between activities. Anomalous activities that were discovered included a truck leaving the dock with its door open and an unusual number of people unloading a truck. 

Anomalous activity detection was also found in indoor environments. Fuzzy membership functions were used to captured several parameters of indoor human activities (location, time, perceived area of subject) from an omni-directional vision sensor and output a fuzzy determination of how "normal" each activity was \cite{Seki2009}. \cite{Yiping2006} presented a similar experimental setup with an indoor omni-directional vision sensor seeking to detect anomalous activities, but used Gaussian models of time and space to determine whether an activity was normal.

\subsection{Full-body Activities}

While detecting anomalies is useful for some applications, detecting the occurrence of a specific pre-defined activity is important for many applications. In the following reviewed studies, various vision sensors recorded subjects’ full bodies to detect specific activities using a variety of computational intelligence. These works have the goal of detecting a specific set of activities and differentiating the occurrence from other activities. These related works are able in the best case to exploit full third-person views of the subjects which provide a maximal amount of input information about both the subject and surrounding environment.

One such study described a system designed to detect a single activity (or a single \textit{type} of activity) - falls \cite{Banerjee2014}. This work focused on applications in elderly living facilities , where timely detection of falls is critical.  It compared the use of three different vision sensors in the same facility and was shown to be capable of detecting three fall-related activities - being upright, sitting, and being on the floor.

Another two related works were focused on using full body video sequences to detect activities in a medical setting. These environments, such as a trauma unit or patient room, may be very busy with doctors and nurses, and activities may be happening frequently or simultaneously. Because of this, both of the reviewed studies required systems capable of monitoring several subjects at the same time to identify activities \cite{Chakraborty2013, Bloisi2009}. The systems were focused on outputting activity information for a set of multiple actors in the scene, providing a summary or log of activities that may assist in the coordination of medical actions such as performing chest compressions, intubation, or checking pulse.

Many related studies which used full-body views from vision sensors were focused on detecting indoor activities of daily living. While the specific sets of activities varied somewhat from study to study, each of these were deployed in home-like environments and detected daily activities such as sitting on a sofa \cite{Figueroa-Angulo2013}, eating dinner \cite{Yao2016}, or reading a book \cite{ElHelw2009}. The sets of activities are pre-defined in each case (i.e. the system is trained to recognize a specific limited set of activities), and the works all make use of scripted training and test datasets. 

\subsection{Partial-body Activities}

A third class of related work focused on detecting activities using vision sensors with visibility of only part of the subjects’ bodies. These limited views often complicate activity detection since the entire posture of the subject is not able to be known by the system. However, the trade-off usually comes with improved usability, portability, or some other convenience to the user. For example, applications using wearable hardware are better able to integrate into all aspects of the user's daily routine.

A fall detection system based on a wearable camera was described in \cite{Mahabalagiri2013}. This work differed from \cite{Banerjee2014} in that it used only a partial view of the subject’s body, since the camera was worn on the torso. This system was capable of detecting a set of three activities similar to the full-body fall detection work, including sitting down, lying down, and falling.

Several studies applied partial-body views from a vision sensor to detect speaking activities. These studies were concerned with identifying speakers in multi-user or noisy environments, citing use-cases such as human-robot interaction \cite{Yoshida2010, Lim2009} and video calling \cite{Savran2018}. Though the recognition of the audio is important in each of these cases, the vision sensor provides valuable information for detecting the beginning and continuation of a speech activity, based on the movement of the subjects’ mouths. In these studies, the detection of activities would not necessarily benefit from knowing the full-body posture of subjects, and the closely focused view of the face was preferred for the enhanced details.

Similarly, studies were discovered which used partial-body views to detect activities while driving. These studies had a similar goal of detecting drivers’ actions while the car was in motion. These activities only require knowledge of the upper body posture. One study sought to build an understanding of actions that occur as the driver completes maneuvers such as turning or approaching an intersection \cite{Martin2017}. The other study was concerned with detecting both driving activities (steering, operating the shift level) and non-driving activities (eating, using cell phone) \cite{Zhao2012}.

As expected, many reviewed papers described systems which used partial-body views from a vision sensor to detect activities of daily living. In these cases, the vision sensor was attached to the subject, either a human or a robot, and computational intelligence was applied to make a determination of the current activity being performed \cite{Li2016, Wu2007}. Due to the positioning of the vision sensors, these studies have only a partial-body view of the subject. Unlike the studies detecting daily activities from full-body views, the partial-body view does not provide the visual information to determine the subject’s complete posture. These studies choose instead to exploit the information of specific objects \cite{Pirsiavash2012, Wu2007, McCandless2013, Nakatani2019} or other people within view \cite{Li2016} to detect the occurrence of an activity.

\subsection{Discussion}
Our results reveal a considerable degree of diversity in the types of activities to which activity detection via vision sensors and computational intelligence has been applied. Overall, the most common type of detected activities were indoor activities of daily living \cite{McIlwraith2009, Wu2007, Yao2016, McIlwraith2008, ElHelw2009, Rowe2007, Figueroa-Angulo2013, Mahabalagiri2013, Ong2013, Li2016, McCandless2013, Nakatani2019}. However, we also found results focused on detecting activities in medical settings \cite{Chakraborty2013, Bloisi2009}, in vehicles \cite{Zhao2012, Martin2017}, in the workplace \cite{Hamid2005}, and interactions with robots \cite{Yoshida2010}. A conclusion may be drawn that the vision and computational tools discussed have broad utility to intelligent systems which must understand the actions of users, regardless of the particular domain, setting, or application of that knowledge.

A drawback to the diversity of the activity domains in our results is that it remains difficult to draw direct comparisons between methods which have seen success. A vision-based system which performs well inside of a home may not perform well when deployed to detect activities in an outdoor work environment. Even within the same activity domain, we did not see a standard dataset emerging to provide a common baseline. In most of the studies, a new dataset was created, and little information was provided on how these datasets were captured. Further, datasets are typically gathered specifically for the evaluation of the system rather than from a truly operational deployment of the system.


%================================================
\section{Types of Vision Sensors}
%================================================
Our analysis revealed that several types of vision sensors have been used for activity detection. These sensors generally fell into the following four categories: 
\begin{enumerate}
    \item traditional fixed RGB vision sensors,
    \item fixed RGB-depth vision sensors,
    \item wearable RGB vision sensors, and
    \item wearable RGB-D vision sensors
\end{enumerate}
    
In some cases, several different devices were used either in concert or separately to provide the best results. The following subsections discuss the use of each of these types of hardware.

\subsection{Fixed RGB Vision Sensors}

Traditional RGB cameras come in a variety of styles, configurations, and sizes, and have been a frequent choice for vision-based activity recognition applications. The vast majority of related work incoporated one \cite{Yoshida2010, Savran2018, Beleznai2012, Wu2007, Seki2009, Yiping2006, McIlwraith2008, ElHelw2009, Rowe2007, Lim2009, Chakraborty2013, Zhao2012} or more \cite{McIlwraith2009, Bloisi2009, Martin2017, Hamid2005} fixed (i.e. non-mobile) RGB vision sensors. In each of these studies, the vision sensors were statically deployed in the environment within which the recognized activity is taking place, providing a third-person view of the subject(s). Within the category of stationary RGB cameras we see variation amongst the types and number of devices used, depending on the specific challenge being addressed.

One challenge that researchers face when using fixed (stationary) RGB vision sensors is a limited view of the environment, either due to a limited view angle or due to occlusion by objects. However, some researchers have been able to alleviate this issue. For example, some studies used multiple cameras simultaneously to provide a more complete view of the monitored environment. In \cite{McIlwraith2009}, a camera network was used to provide a comprehensive view of several seating stations within one room. The system mimics a home environment where activities such as dining, studying, and reading may be expected to occur in distinct locations. Individual camera nodes may provide sub-decisions, but communication between all nodes leads to a final result. Other studies that used multiple cameras examined differing use cases. For example, in \cite{Hamid2005} multiple cameras were deployed with partially overlapping viewpoints in order to provide complete coverage of activities that transpired in a loading dock. This allowed detection of multiple activities occurring in parallel at a given time.  Multiple cameras were also used in \cite{Martin2017} to provide detailed images of the face and hands of a subject driving a vehicle to analyze the coordination of the head and hands while maneuvering the vehicle. The cameras were positioned to optimally capture the relevant areas of the subject while excluding irrelevant details such as the dashboard or the passenger section of the car.

Our investigation also showed that researchers have found it possible to achieve an enhanced perspective without the use of multiple cameras. Two related studies incorporated omni-directional vision systems into the design of their activity recognition systems \cite{Seki2009, Yiping2006}. These cameras map a complete 360 degree view onto a typical two-dimensional intensity image, providing full monitoring of an area with a single hardware device, typically mounted overhead in the center of the space. Though this results in a distorted image due to the 360-degree perspective, the image may still be processed and human postures are still reliably discernable \cite{Seki2009}. A similar single-camera experimental setup is shown in \cite{Yiping2006} to effectively monitor an entire panoramic view of a room. The omni-directional vision device is capable of monitoring activities occurring in a seating area and four areas of entry or exit to other rooms.

A challenge in detecting activities with fixed RGB vision sensors is the lack of depth information from the scene. That is, only two dimensions of data are recorded, leaving out potentially valuable information about the movement and positioning of objects. However, traditional RGB cameras (which do not directly record depth information) can recover approximate depth information when used in pairs as a stereo camera. A depth dimension may be constructed by comparing the perspective of each camera given the predetermined distance between them. Two of the studies included in our review take this approach \cite{Lim2009, Bloisi2009}. In \cite{Bloisi2009}, depth information from a stereo camera is exploited to detect the background in each frame and subsequently remove it, leaving only foreground objects for consideration in activity recognition. Depth information from stereo cameras has also been used to determine distances between multiple subjects in the same area \cite{Lim2009}. The precision afforded by the depth information allows for improved facial tracking and differentiation between subjects in closer proximity to one another, improving the overall ability to differentiate the active speaker.

\subsection{Fixed RGB-Depth Vision Sensors}

As previously discussed, traditional RGB vision sensors record only a two-dimensional mapping of the three-dimensional environment. This is an important limitation because it is not possible to reconstruct the third dimension with high accuracty without the use of a standard vision system that integrates multiple vision sensors to provide different perspectives of the same area (i.e. a stereoscopic camera).

An alternative approach to providing depth information from a stereoscopic RGB vision sensor system is to use an RGB-Depth vision sensor. These sensors incorporate a traditional RGB camera, an infrared (IR) camera, and an IR projector. The IR projector projects a pattern of IR light onto the scene which is recorded by the IR camera to determine depth. The depth measures are combined as an extra channel of information with the image captured by the RGB camera. Several studies within our sample used RGB-D vision sensors \cite{Banerjee2014, Yao2016, Akbari2017, Zhao2017, Figueroa-Angulo2013, Ong2013}.

The systems described in \cite{Ong2013, Yao2016} detect activities based on sets of joint features describing the flexion, extension, rotation, or position of a variety of human joints. The additional depth information from the vision sensor is required in this case to provide the necessary inputs to the recognition algorithms. \cite{Figueroa-Angulo2013} describes a similar approach where digitized three-dimensional skeletal and joint models are extracted from the depth images captured by the sensor and then used to detect activities based on a statistical Markov model. A multi-modal sensing system built a human motion model in \cite{Akbari2017}. An RGB-D vision sensor is paired with wearable measurement devices (IMU, gyroscope, accelerometer) to observe motions, and determine situations in which each device is better-suited to capture the data.

We also found that depth information may benefit systems that need to distinguish between multiple subjects. An RGB-D vision sensor was used in \cite{Zhao2017} to provide biometric identification of users based on bone lengths. The authors acknowledged that while previous work indicates that fingerprints or iris scans are more reliable means of identification, the model built by an RGB-D sensor is much more user-friendly and allows passive user identification from a distance, which may be important in a dynamic multi-user environment such as a workplace.

In addition to the extra dimension of positional input data, another convenience provided by the use of an RGB-D vision sensor is improved performance in poor lighting conditions. While RGB vision sensors may have difficulty properly adjusting exposure and maintaining a clear image in bright or low light, the IR projectors and cameras in RGB-D vision sensors provide a consistent result in most lighting conditions since they are not susceptible to changes in visible light and have an integrated source of IR light to illuminate the scene. \cite{Banerjee2014} demonstrates this and gives a comparison of human silhouette extractions using a webcam with an IR filter and an RGB-D vision sensor in an eldercare facility. The RGB-D sensor is shown to be effective, though the depth information begins to degrade as the subject moves farther from the vision sensor (due to natural scattering of IR light from the projector over this distance).

\subsection{Wearable RGB Vision Sensors}

A drawback to all of the RGB and RGB-D vision sensors discussed to this point is that they are not portable. They only record data within a fixed environment, which may not be sufficient for all use cases. For example in an indoor environment, if the person moved away from the room in which the sensor was placed, the system would fail to raise an alarm if the person fell down. Since many rooms may pose a high risk of falls (bedrooms, bathrooms, staircases, kitchens), providing full coverage is difficult. For this reason, many related works use wearable sensors to enable systems to detect and recognize activities wherever the user travels.

Many examples can be found which incorporate wearable vision sensors \cite{Radhakrishnan2016, Mahabalagiri2013, Li2016, Baraldi2015, McCandless2013, Wang2018, Damen2018ScalingDataset, Sudhakaran2018, Gokce2019HumanPairs}. In all of these, the vision sensors were placed on the subject to provide a first-person view of the activity being performed. In many cases, this meant that features describing the posture and joints of the subject were no longer available because they were out of view. However, the first-person view provides an opportunity to identify objects that the wearer interacted with \cite{Pirsiavash2012, Radhakrishnan2016, McCandless2013, Damen2018ScalingDataset}, people with which the wearer interacted \cite{Li2016}, areas of visual focus of the subject \cite{Sudhakaran2018, Matsuo2014}, and estimated motion of the subject based on camera motion \cite{Mahabalagiri2013}. 

In \cite{Radhakrishnan2016}, a system was built to identify the action of picking an object off of a shelf in a store. The action event was detected via non-vision (inertial) sensors, but the identification of the specific object was accomplished by a wearable camera worn on the user's wrist. This illustrates how wearable vision sensors may be used to narrow the focus of the input data by intentionally choosing the placement of the camera on the subject.

In other cases, the wearable vision sensor can provide information about specific hand gestures instead of objects with which the subject is interacting. This is demonstrated in \cite{Baraldi2015}, where a vision sensor embedded in a pair of glasses provided a view of the subject’s hands. Hand gestures were recognized and the gesture was applied to the object in the center of the subject’s gaze. The glasses and hand gestures presented a natural means of interacting with the system which would not be achievable with a stationary vision sensor.

A final use case we found for wearable vision sensors was to describe the motion of the subject wearing the device. When mounted in a fixed position on the body, movement recorded by the sensor across temporal image sequences may be attributed to motion in that part of the body, supplying information about both the environment in front of the subject as well as the posture of the subject without the subject being directly in the field of view. In one study, an RGB sensor was worn on the subject’s trunk and the movement of the camera was used to estimate the movement of the torso, enabling the detection of sitting and lying down without any part of the torso in view \cite{Mahabalagiri2013}. This is a good example of how wearable sensors may be used to capture derivative information from the scene, in addition to information of objects or subjects that are directly within the view of the sensor.

\subsection{Wearable RGB-D Vision Sensors}

Very few studies are found which employed wearable RGB-D vision sensors for the purpose of activity recognition. The approach described in on study was based on knowleddge of object interactions, and a robot (rather than a human) was equipped with the wearable RGB-D vision sensor \cite{Li2016}. The purpose of the study was for the robot subject to become aware of the activity it was performing by using knowledge of nearby objects. The wearable sensor was used to detect the objects interacting with the robot’s hands, and that data was fused with position and joint features provided from the robot’s operating system, taking advantage of the availability of both object and posture data. 

\subsection{Secondary Sensors}

In our survey of related work, secondary non-vision sensing devices were soemtimes used to enhance the data captured by the vision sensor \cite{Yoshida2010, Radhakrishnan2016, Savran2018, Wu2007, Akbari2017, McIlwraith2008, Zhao2017, Lim2009, Li2016}. Multiple studies used a microphone to assist in voice activity detection along with the vision sensors \cite{Yoshida2010,Savran2018,Lim2009}. The vision sensor was important to these studies for identifying the speaker in the observed space, and the microphones were used to provide auditory data which can provide extra confidence in the number of voices occurring at a given time. Other reviewed studies made use of smart watch devices to provide  wrist motion features as input to computational intelligence methods \cite{Radhakrishnan2016, Zhao2017}. In \cite{Wu2007}, radio frequency identification (RFID) tags were applied to objects and the RFID device was used to correlate vision sequences with proximity to objects in the environment. \cite{McIlwraith2008} combined an ear-worn accelerometer with a vision sensor and extracted non-overlapping discriminatory features from each device to detect activities. Similarly, \cite{Akbari2017} paired a vision sensor with an IMU device. The IMU device provided finer details of the motion the subject performed, allowing the activity detection to be fine-tuned based on the speed of the movement. These studies demonstrate situations in which multiple sensing devices and/or different sensing modalities may combine to provide a richer dataset for activity detection.

\subsection{Discussion}

In our survey of related work, we found a variety of vision sensors, including both fixed and wearable RGB and RGB-D sensors. We found that there were significant variations on the use and implementation of each of these, including special cases such as omni-directional or stereoscopic sensors. Our results indicate that each of these pieces of hardware are capable of providing positive results in activity detection when used with computational intelligence, though the choice of hardware requires careful consideration of the objectives of the system. 

We note that the choice of vision sensor is largely dependent on two factors - 1) the scene in which activities will take place and 2) the types of activities that need to be detected. Wearable sensors are frequently chosen when the scene is not a fixed location (i.e. the activity may occur in any location in which the user is present) \cite{Baraldi2015} or when objects within the user’s perspective are more important than the overall posture or positioning of limbs and joints \cite{Baraldi2015, Li2016, Radhakrishnan2016}. When choosing between RGB and RGB-D sensors, RGB-D sensors are a common choice when differentiation of activities seemingly necessitates precise three-dimensional data \cite{Ong2013, Figueroa-Angulo2013} or when lighting conditions may affect the performance of visible-light-based RGB sensors \cite{Banerjee2014}.

%================================================
\section{Computational Intelligence Methods}
%================================================

Our analysis of related work indicated that a wide variety of computational intelligence methods have been used with vision sensors for the purpose of activity detection. We categorized studies by algorithm and found the use of both supervised and unsupervised methods. In general, supervised methods are a frequent choice when the desired output from the system is an explicit activity label from a set of predefined activities. Applications which need only to detect the abnormality of an activity often turn to unsupervised intelligence methods since these methods do not require the manual annotation of truth data, which can be time consuming. We review both methods in relation to the use of vision sensors and activity detection below. 

\subsection{Stochastic Modeling}

Many of the reviewed studies employed some variant of a stochastic process model \cite{Rowe2007, Figueroa-Angulo2013, Lim2009, Chakraborty2013}. The activity detection capability of a multiple vision sensor network described in \cite{Rowe2007} was demonstrated by a Markov model showing regions of activity across multiple rooms with transition probabilities determined by a collected dataset. In another case, a large composite Hidden Markov Model (HMM) comprising smaller HMMs was built in \cite{Figueroa-Angulo2013}. The smaller HMMs individually specified the states within just a single activity and contained transitions back to a common initial and final state. The larger HMM was composed by allowing transitions between the common initial state, which represented transition between the smaller individual activities. The HMMs were trained on features representing the skeletal structure of the observed subject. In a different application of the Markov property, a Markov logic network (MLN) was employed in \cite{Chakraborty2013}. The MLN was driven by an activity grammar that described steps within an activity, and formed relationships between predicates and objects (e.g. “stethoscope approaches patient”). A Langevin process model, rather than a Markov model, was used in \cite{Lim2009}. This work fused visual and audio data to determine speaking activities and identify speakers in a multi-user environment. 

\subsection{Fuzzy Logic and Clustering}

Activity detection methods based on fuzzy logic were also used in several of the reviewed studies \cite{Banerjee2014, Seki2009, Yao2016, Akbari2017}. Fuzzy methods introduce degrees of truth, rather than binary decisions. Fuzzy C-Means clustering was used in \cite{Yao2016} to turn training posture feature vectors from an RGB-D vision sensor into fuzzy rules to support classification via type 2 fuzzy logic. An extension of Fuzzy C-Means clustering, Gustafson-Kessel clustering, was used to cluster silhouette and image moment features to recognize activities using RGB-D sensors \cite{Banerjee2014}.

In addition to fuzzy clustering, we also found other applications of fuzzy logic. Inputs from an omni-directional vision sensor were recorded in terms of fuzzy variables in \cite{Seki2009}. The use of fuzzy variables improved the performance of the activity detection around the discretizing borders when compared with previous work using Bayesian methods \cite{HirokazuSeki2008}.

\subsection{Bayesian Methods}

Our survey also included studies that used Bayesian methods to detect activities \cite{McIlwraith2009, Yoshida2010, Wu2007}. \cite{McIlwraith2009}, which employed a network of several vision sensor nodes, proposed a greedy structure learning algorithm based on the Bayesian Information Criteria (BIC). Each sensor node built a Bayesian network to learn activities for which the node was a candidate for detecting. Candidate nodes contributed data to the network to make a final determination of an activity. The distributed approach allowed sub-decisions to be made at the node level, which positively impacted the fault tolerance of the network.

A Dynamic Bayesian Network (DBN) was used in \cite{Wu2007} to automatically acquire models of activities and objects that are involved with them. The DBN was formed from internet-based knowledge repositories, object information, visual frames, and RFID inputs. Learning was done in an unsupervised manner so that data did not need to be labeled (except for the purpose of testing).

A third study used a Bayesian network to perform voice activity detection from a combination of RGB video and audio data \cite{Yoshida2010}. The inputs included the log-likelihood of silence in audio data calculated by a speech decoder, a feature based on the height and width of the lips as captured by the vision sensor, and the confidence that a face was detected in the visual data. Probabilities for the Bayesian network were obtained from a Gaussian Mixture Model (GMM) trained in advance on a separate set of training data.

\subsection{Gaussian Mixture Models (GMM)}

Gaussian modeling, another probabilistic method, was also used in the examined studies \cite{McIlwraith2008, ElHelw2009, Yiping2006}. In \cite{McIlwraith2008, ElHelw2009}, GMMs were used to perform fusion of data from multiple sensors for the detection of activities. The models were based on data provided by a vision sensor and an ear-worn accelerometer. Vision sensors provided features such as bounding box aspect ratio and eigenvectors of silhouettes, and average optical flow within the blob, while the wearable device provided information on the tilt and movement of the head. \cite{Yiping2006} used Gaussian distributions to build spatial and temporal models of activities. Using the combination of these models, activities were determined to be either normal or abnormal. Recorded activity information was then used to update the models. 

\subsection{Neural Networks}

Deep learning and neural networks were used for activity detection in multiple of the reviewed studies \cite{Li2016, Savran2018, GillaniFahad2013LongHome, Lu2019, Li, Wang2018}. Multiple types of Recurrent Neural Network (RNN) structures were used in \cite{Li2016}. These structures were shown to be well-suited to performing classification based on sequential video frames. Convolutional layers of the networks operated on pre-processed visual frames which were later combined with joint features as input to a Long Short-Term Memory (LSTM) layer. The LSTM layer determined when a tracked hidden state should be updated, contributing to the strong performance of the system against the temporal video data. Similar positive results were shown against temporal data in \cite{Savran2018}, where audio data was input to deep learning detectors after visual sensor data was used to detect speaking activities. Speaking activity was visually detected by analyzing motion near the mouth.

\subsection{Other Methods}

The previously discussed methods account for a majority of the computational intelligence methods found in the examined studies. However, a variety of other methods were also used to detect activities using vision sensors. A simple decision tree was used in \cite{Radhakrishnan2016} to detect the single activity of picking an item from a shelf. In another study, the K-means clustering algorithm was used to cluster unlabeled skeletal features into differentiable activities \cite{Ong2013}.

A more complex density-based clustering algorithm was used in \cite{Beleznai2012} to detect rare activities using binary space-time descriptors. Two of our included studies used a Random Forest classifier to detect driving activities \cite{Zhao2012, Martin2017}. These were the only studies in our results to use this method. 


\subsection{Discussion}
We found that numerous computational intelligence methods have been used with vision sensors to detect activities. As with the choice of vision sensor, this may be driven by the set of activities that need to be detected. In applications requiring only a measure of normality for an activity (e.g. normal vs. abnormal) rather than a specific classification of the activity in a given set, clustering methods are shown to be robust \cite{Beleznai2012}. In applications where activities of daily living are detected, we see several distinct methods including Bayesian methods \cite{Wu2007, McIlwraith2009}, Stochastic modeling \cite{Rowe2007, Figueroa-Angulo2013}, and Gaussian Mixture Models \cite{ElHelw2009, McIlwraith2008}. The differences in the experimental setup of these similar studies present a difficult comparison of methods against each other. 

A commonly cited factor in the success of computational intelligence methods in classifying an activity into a predefined set of possible activities is the ability for the method to take into account the various states within an activity. Activities typically have both spatial and temporal components, and our results indicate strong performance with methods that can either consider time-aggregated data, have a memory of previous outcomes, or in some way indicate change of pose or environment over time. Examples of this are seen in neural network and deep learning methods \cite{Savran2018, Li2016}, stochastic modeling methods \cite{Rowe2007, Figueroa-Angulo2013, Chakraborty2013, Lim2009}, and Bayesian methods \cite{Wu2007}.

Comparing computational intelligence methods and the results associated with each was complicated by the lack of consistency in reporting results. Several of the reviewed studies failed to report an accuracy metric or the details of the datasets that were used.

%================================================
\section{Key Takeaways}
%================================================
The survey of related work reveals trends in the hardware, computational methods, and experimental approaches used to detect, classify, and describe human activities. There are a wide variety of methods applied in different combinations to accomplish very distinct tasks, and the field appears to suffer from inconsistency in experimental methods and reporting of results. However, there are clear common limitations that are identified:
\begin{enumerate}
    \item The use of black-box methods such as convolutional neural networks limits the generalizability of the related works. Many reviewed publications describe systems built for very specific purposes and environments, but do not classify a set of activities that is encompassing of instrumental daily activities.
    \item Most of the surveyed methods offer very little in the way of interpretability of the learned input features. As a result, classification of activities is either completely correct or completely incorrect; an unrecognized activity is given no additional context or description by these systems. If input features are able to be interpreted, then some context may be discerned about the activity even in the event of an incorrect classification.
    \item Related works acknowledge that complex activities are often defined as the composition of smaller activities in sequence, but do not address the detection of activities at these various levels.
\end{enumerate}

%================================================
\chapter{Methods}
%================================================

This chapter describes the methods that are used throughout the experiments.

%-------------------------
\section{Image Features}
%-------------------------
Image features computed from first-person video form the basis for the activity detection system described in this document. These features provide the mechanism by which pixels in the two-dimensional RGB video are transformed into data and knowledge leveraged by the computational methods. 

\subsection{Optical Flow}
Optical flow is an algorithm that takes two consecutive frames of video (or frames that occur very closely in time) as input and provides an estimate of the motion vector for pixels in the frame. The motion estimation can be computed for every pixel in the input (\emph{dense}) or for a specific subset of pixels which have been identified as being important (\emph{sparse}). The motion estimates for pixels are computed by considering the linear displacement of corresponding pixels in the two frames as shown in Figure \ref{fig:optical_flow}. The optical flow algorithm provides an estimation of the vector $<dx, dy>$.

\begin{figure}
    \centering
    \includegraphics[width=\linewidth]{figure/optical_flow.png}
    \caption{The pixel specified in the first frame at time $t$, $I(x, y, t)$, translated by $(dx, dy)$ units and is located at $I(x+dx, y+dy, t+dt)$ in the next frame at time $t+dt$.}
    \label{fig:optical_flow}
\end{figure}

To compute optical flow, it is necessary to assume that pixels in the image $I(t)$ have equal intensity in the image $I(t+dt)$ for small values of $dt$. This is valid in practice because a small $dt$ represents a time difference where the appearance of the object and lighting conditions are unlikely to change drastically. Based on this, we can set the equation for each pixel in the two images as equal (known as the constant intensity assumption): $I(x, y, t)  = I(x+dx, y+dy, t+dt)$. This equation can be solved for the resulting motion vector using polynomial expansion, Taylor series approximation, and a variety of other methods \cite{Farneback2003Two-FrameOn, Horn1980DeterminingFlow, Lucas1981AnVision}.


%-------------------------
\section{Fuzzy Inference Systems}
%-------------------------
Fuzzy Inference Systems (FIS) are systems that accept a series of inputs, process them according to predefined rules, and produce an output. The variables in a fuzzy system, including inputs and outputs, are called fuzzy variables, and take on linguistic values rather than precise numeric values. For example, in a fuzzy system, a temperature variable may be defined to take values in the set ['cold', 'cool', 'warm', 'hot']. Each variable is defined over a \emph{universe}, or range of numeric values. 

\begin{figure}
    \begin{subfigure}{.5\linewidth}
        \centering
        \includegraphics{figure/mem_func.png}
        \caption{Example membership functions for the fuzzy variable temperature}
        \label{fig:mem_func}
    \end{subfigure}
    \begin{subfigure}{.5\linewidth}
        \centering
        \includegraphics{figure/fuzzification.png}
        \caption{Fuzzification of the crisp value 10 degrees to a fuzzy temperature value}
        \label{fig:fuzzification}
    \end{subfigure}
    \newline
    \begin{subfigure}{.5\linewidth}
        \includegraphics{figure/defuzzification.png}
        \caption{Defuzzification of a fuzzy temperature value to 22 degrees.}
        \label{fig:defuzzification}
    \end{subfigure}
    \caption{Illustration of membership functions, fuzzification, and defuzzification process.}
    \label{fig:mem_fuzz_defuzz}
\end{figure}

Membership functions exist for each linguistic term in the set to define a mapping of numeric values to the linguistic terms. Figure \ref{fig:mem_func} illustrates membership functions for the example temperature variable. It is often the case that membership functions overlap. In the overlapping region we can say that the value has membership in multiple terms. This is a critical distinction for fuzzy systems and allows a variable to express some uncertainty. For the example of the temperature variable, a value of 10 degress Celsius falls in both the 'cold' and 'cool' membership functions as illustrated by \ref{fig:fuzzification}. 

The process of transforming the crisp numeric value to the fuzzy value is referred to as \emph{fuzzification}. The reverse process, transforming a fuzzy value to a crisp value, is appropriately called \emph{defuzzification} and is illustrated in \ref{fig:defuzzification}. There are several common defuzzification methods, including centroid, maximum membership, and bisector of area. Defuzzification is an important step in determining the final output of the fuzzy system. 

In this work, we use only Mamdani-type inference in which all output variables are expressed as fuzzy sets. Rules in a Mamdani inferences system are expressed in the form ``\emph{IF $X$ is $x_i$ and $Y$ is $y_j$ THEN $Z$ is $z_k$}'', where $X$, $Y$, and $Z$ are each fuzzy variables and $x_i$, $y_j$, and $z_k$ are terms belonging to the fuzzy sets. An example in a fuzzy system controlling a thermostat might be ``\emph{IF outsideTemp is cold and indoorTemp is low THEN heatSetting is high}.'' Rules may be conjunctive or disjunctive and may contain any number of operands in the rule.

During the evaluation of a rule, the level of membership of the input terms are considered to determine the strength of membership of the output. The Fuzzy `and' operator results in the minimum level of membership strength between the operand terms being passed to the output. By contrast, the fuzzy `or' operator results in the maximum level of membership strength between the two operands being passed to the output.

To illustrate the evaluation of rules, consider the previous example ``\emph{IF outsideTemp is cold and indoorTemp is low THEN heatSetting is high}.'' Let $outsideTemp['cold'] = .35$ and let $indoorTemp['low'] = .1$. The rule is evaluated as $heatSetting['high']=min(.35, .1)=.1$. The final output of the fuzzy system depends on the evaluation of the rest of the rules in the system and the defuzzification method used.





%================================================
\chapter{Experiments and Results}
%================================================

This chapter describes the experiments completed to resolve the research questions posed in the Introduction.

%================================================
\section{Gait Speed Comparison from Wearable Camera and Accelerometer Vest in Structured and Semi-Structured Environments}
%================================================
To address our first research question \emph{(Can wearable sensors be used to unobtrusively monitor activities of subjects in a semi-structured environment?)}, we limit the scope of activities to a single activity - gait. By fixing the activity, we focused instead on the ability of various sensors to describe the parameters of the activity in semi-structured environments (i.e. home-like, controlled, indoor setting without the noise of environmental motion found in public settings). \emph{This work was published in \cite{Schneider2019ComparisonEnvironments}. A slightly modified version is provided below in this section for reference in this draft.}

Gait analysis is an area of research that has seen an increasing focus due to its applicability to a wide range of age-related health issues which may impact our growing elderly population \cite{Ortman2014AnStates}. For example, Beauchet et. al found evidence that dementia can be predicted by poor gait performance \cite{Beauchet2016PoorMeta-Analysis}. Similarly, Valkanova and Ebmeier show that evidence strongly supports a relationship between gait and impairment of cognitive functions in patients with mild cognitive impairment and Alzheimer's disease \cite{Valkanova2017WhatEvidence}. These findings illustrate the potential of using gait analysis in detecting symptoms of age-related illnesses.

While many objectively quantifiable gait parameters could be used for effective decision support and automated monitoring, the simple measure of \textit{gait speed} has shown to be an accurate predictor of mobility, health, and even mortality \cite{Afilalo2010GaitSurgery, Viccaro2011GaitForce, Fitzpatrick2007AssociationsPersons}. Gait speed is thus a critical parameter for evaluating the utility of candidate gait analysis systems. Therefore, the goal of this study was to determine the feasibility of using our system, comprising an affordable and non-invasive wearable camera and computer-vision based processing methods, to classify gait speed of healthy individuals. Samples of gait were collected at three self-determined over-ground walking speeds (slow, medium, fast). Since accelerometer-based methods have been successfully used to quantify gait, we deployed an accelerometer near the subject's right hip for the purpose of providing a direct point of comparison for our system to a more widely used device. We also compared the capabilities of both devices to a research-grade optical motion capture system which represents the ``gold standard'' for gait analysis. The comparison to both another wearable system and a high-precision standard provides validation for using our single-camera system for gait analysis tasks.

\subsection{Related Work}
Expensive laboratory-based gait analysis systems can provide extremely robust quantification of human gait and locomotion. For example, highly precise three-dimensional motion capture systems have recently been used to study detailed gait features across age groups in healthy individuals \cite{Chien2015TheIndividuals} as well as individuals with Parkinson's disease \cite{Corona2016QuantitativeDisease} and Alzheimer's disease \cite{Rucco2017Spatio-temporalCapture}. While these systems can often provide an in-depth description of gait, they are not feasible for the use-case of continuous monitoring due to their size, complexity, and cost. Continuous monitoring of patients in their natural environments during everyday activity provides a more constant and natural sampling of gait activity, enabling the detection of changes in performance over time. Additionally, user-friendly and lower cost pervasive health monitoring systems reduce the burden on patients to make trips to a physician's office, which may be especially cumbersome for the elderly.

Recent work has focused on providing convenient, in-home solutions for activity recognition, eliminating the need for a laboratory setup. Video-based methods may offer inexpensive solutions with performance similar to more sophisticated motion capture systems or floor sensors \cite{Wang2013TowardAdults}, but suffer from obstructed line-of-sight within the home environment. Audio-based systems have also been used to analyze gait in indoor environments \cite{Geiger2013Gait-basedFeatures, Altaf2015AcousticSounds}. While both audio- and video-based systems have shown promise, their use is limited to a specific preconfigured location. To overcome these issues, gait analysis is also being performed with wearable devices, such as smart watches \cite{Suh2016Kalman-Filter-BasedSmart-Watch}, shoe-based wearable sensors \cite{Mariani2013On-shoeDisease}, and wearable accelerometers \cite{Fortune2014ValidityVelocities, Hartmann2009ConcurrentAdults, DelDin2016ValidationUse, Chung2012GaitAccelerometer}. By instrumenting the subject instead of the environment, the systems become portable and problems such as line-of-sight obstruction can be avoided. While these solutions provide promising results for gait analysis, they tend to be either still too complicated for in-home use (e.g., due to the number of components), or incapable of matching the level of precision and capturing gait performance as comprehensively as  laboratory-grade motion analysis systems. Thus, there are trade-offs between accuracy and cost for laboratory-grade motion analysis systems and in-home wearable systems for gait analysis.

Our system uses a single head-worn camera to collect first-person video. Using computer vision techniques, we are able to extract optical flow output from the video that mimics the ability of a low-resolution accelerometer to register movement parameters \cite{Schneider2017PreliminaryProcessing}. A benefit of a vision-based sensor such as our system is that the video data can provide additional context to the in-home monitoring scenario. For instance, if an unexpected event occurs with respect to gait speed, the monitoring system could notify additional automated or manual review processes to analyze the specific related video segment and determine whether the event was a clinically significant event, such as a fall, or simply an abrupt stop. It may also be possible to analyze the coarse direction of the subject's visual attention while walking and to identify objects that are being interacted with or other factors that may impact gait performance. Methods based entirely on accelerometer or pressure sensors are unable to explain variations or interruptions that are seen in daily gait activities and would not be able to inform further analysis processes. While vision-based in-home monitoring systems may raise privacy concerns by recording subjects and others in their home, automated methods of processing the video upon recording would eliminate the need to store the raw video, mitigating the privacy risk. The storage or transmission of computed motion-based features removes nearly all identifying information as these features are essentially equivalent to those recorded by inertial sensors, which do not raise the same concerns.

Our previous work based on the use of a wearable camera (and accelerometer for comparison as a well-established device) has involved collecting data from subjects on a treadmill \cite{Schneider2017}. However, limiting the data collection to occur on a treadmill was an artificial limitation, especially for the in-home use-case being described. In this paper, we remove this limitation and also investigate the impact of changing this aspect of the experimental design to incorporate semi-structured, natural gait sequences from nineteen participants.

\subsection{Experimental Methods}
In this section we describe our experimental methods, including our data collection designed for this experiment and our use of signal processing and filtering to process the collected dataset.

\subsubsection{Data Collection}

\begin{figure}[!t]
\centering
\includegraphics[width=1.8 in]{figure/MoCapRoom}
\caption{The motion capture laboratory in which data was collected from participants}
\label{fig_mocap}
\end{figure}

\begin{figure}[!t]
\centering
\includegraphics[width=1.8 in]{figure/sensorAlign}
\caption{Placement of Sensors on Subject and Alignment of Axes Between Sensors}
\label{fig_axes}
\end{figure}

Accelerometer, motion capture, and first-person video data were collected from nineteen participants as they walked overground six times over a distance of four meters in a large motion capture laboratory, free from any physical obstructions such as furniture or walls (Figure \ref{fig_mocap}). The participants were healthy college students ranging from 18 - 21 years old. Ten of the participants were male and the remaining nine were female. Participants were instructed to walk at three self-determined speeds: slow, medium, and fast. Categorical speeds were used for multiple reasons. First, as participants were walking over the ground, and not on a treadmill, it would have been difficult to adequately control gait speed. Second, the time series data from inertial sensors and the processed video features indicated the frequency of each subject's gait. Estimating the continuous gait speed requires knowledge of the exact stride length of a subject. Categorical gait speed was thus more appropriate as slow, medium, and fast gaits naturally correspond to an increasing step frequency regardless of stride length or distance covered.

Each subject wore two commercial devices during data collection (see Figure \ref{fig_axes}). The Pivothead SMART Architect Edition glasses \cite{Pivothead2017Pivothead} were used to record video of the activity in HD resolution (1920 x 1080) at 30 frames per second. The device is worn as a pair of eyeglasses, and the camera is located in the center of the glasses, above the nasal bridge, aimed directly forward. The glasses are nearly indistinguishable in shape and weight from a normal pair of glasses, providing a comfortable and natural sensor that is easily integrated into daily routine with no encumbrance or health risks to the wearer. The Hexoskin smart shirt \cite{CarreTechnologies2017HexoskinShirts, Banerjee2017ValidatingManagement} was also worn, providing tri-axial accelerometer readings at 64 Hz (data from the remaining sensors in the Hexoskin shirt were not analyzed for this study). The accelerometer was located near the right hip on the torso of the subject. Gait data were also recorded with a 20-camera Motion Analysis Corporation Kestrel motion-capture system at a sampling rate of 120 Hz and motion data were processed using Cortex v. 6.2 software (Motion Analysis Corp., Santa Cruz, CA). Each participant was instrumented with motion capture markers according to the Cleveland Clinic marker set. This model includes markers tracking the position of the feet, legs, trunk, arms, and head. While the position of each marker was recorded, our analysis focused on the head marker (for comparison to the head-worn glasses results) and estimates of whole-body center of mass derived from the global marker set using a whole-body mass model calculated from Zatsiorsky-Seluyanov's body segment inertia parameters \cite{deLeva1996AdjustmentsParameters}. Each of the three systems independently and simultaneously recorded the gait sequences that were performed.

The Pivothead glasses were purchased for \$300 USD and the Hexoskin vest and device may be purchased together for \$499 USD, making them easily available to consumers. While the exact price of the motion capture system is not immediately available and will vary based on configuration, the cost of the 20-camera system is roughly \$100,000 USD. A minimal set of four lower-precision cameras could be obtained for less than \$10,000 USD. Even considering the lower-cost motion capture option, the consumer-grade devices have the advantage of being easy to use, while the motion capture system requires an expert user and extensive instrumentation of the subject. While the motion capture system presents an excellent means for collecting our high-precision truth data, the cost and complexity imply that the motion capture system will only be feasible in a controlled clinical laboratory, and not for continuous, in-home monitoring. While we did not investigate real-time processing, the computational requirements for data from the single camera and inertial sensor would also be much lower than the 20-camera system.

\begin{figure*}
\centering
\includegraphics[width=7in]{figure/meanplots}
\caption{Mean Plots with Standard Error Bars for Recorded Features By Gait Speed}
\label{fig_meanPlots}
\end{figure*}

\subsubsection{Signal Processing}
The intent of incorporating the glasses-worn camera into the experiment was to use the collected video to describe the subject's head motion in two dimensions (the frontal and vertical planes) throughout the gait sequence. Since the camera faces directly forward from the glasses, the frontal and vertical axes in physical space correspond to the x- and y-axes of the recorded video. While the camera device collects less data at a lower sampling rate and spatial resolution than a highly accurate 3-D motion capture system, the device is extremely portable, affordable, and simple to operate. However, the visual data requires processing in order to extrapolate information about the movement of the subject who is not in the view of the camera. The Lucas-Kanade optical flow technique was applied to the collected video samples in order to quantify participant motion from the videos \cite{Lucas1981AnVision}. This method output a displacement vector for a series of significant keypoints within a given video frame. An average vector was computed from all keypoint vectors per frame to find a single two-dimensional vector which represented the overall displacement in each frame of video. This approach was previously validated against other possible computer vision techniques and was found to provide the most accurate representation of the actual displacement between frames \cite{Schneider2017PreliminaryProcessing}.

The two-dimensional (frontal, vertical) components of the optical flow displacement vectors were considered over time to generate two separate sets of time series data. Three-axis time series data were also collected from the body-worn accelerometer, and head and center of mass data (each in three dimensions) from the motion-capture system were also analyzed. The time series data were manually separated into segments collected during each of the six trials. The collected data were manually segmented by examining the video and audio, and then recording the start and stop times of each gait segment within the video on a per-frame basis. None of the systems directly provide a determination of when a gait sequence occurs, though it would be feasible to automate this detection based on the collected features and the video data. For this initial work, such a system was not developed since the focus was the output of the system during gait activities. Motion-capture data were filtered in Cortex using a 4th-order low-pass Butterworth filter with a cut-off frequency of 6 Hz, which is the default setting for the low-pass filter in Cortex. Cut-off frequencies in motor control research generally range from 6 to 10 Hz, depending on the behavior being observed. Given that the observed behavior was walking, 6 Hz was much higher than the frequencies of interest and only filtered out sensor noise.

Because the wearable devices are commercial devices which operate independently of each other, it was not possible to guarantee synchronization of the data collection. This precluded any direct comparison of the raw time series data from each of the sensing devices, since errors in synchronization of the collected data would negatively impact any calculations. However, it is not necessary to directly compare the time series data - gait speed alone has been shown in clinical applications to be a predictor of cognitive disorders such as dementia \cite{Bramell-Risberg2005LowerControls}. To overcome the limitation on direct comparison between the time series data from each device, we moved data out of the time domain and instead derived frequency-based features in the following manner. 

A periodogram calculation was applied over the entirety of each walk segment for each channel of data being considered. As shown in \cite{Schneider2017PreliminaryProcessing}, the periodogram transformation can be used to identify main frequencies that occur in each time series. We identified for each time series the frequency with the highest amplitude from the computed periodogram to serve as our gait metric. While the periodogram provides an analysis of constant gait speed in \cite{Schneider2017PreliminaryProcessing} and a successful indication of gait speed in temporally short gait sequences collected for this study, longer gait sequences containing changes in gait speed may be better analyzed with methods that consider time locality, such as a spectrogram. For this study we have assumed that gait speed is constant in each gait segment.

\subsubsection{Statistical Analysis}
The goal of this work was to determine the feasibility of classifying gait samples categorically by their speed with a specific interest in the performance of the video-based wearable system. We determined whether the collected video-based features were impacted by gait speed in a manner similar to the features from more traditional gait analysis devices. We used analysis of variance (ANOVA) to determine whether our gait metric was significantly impacted by gait speed. Separate ANOVAs were conducted for each plane of motion for each gait measurement system. Data were screened for outliers ($\pm$ 2.5 standard deviations from the median) prior to analysis; 6 trials were identified as outliers, all for the motion capture system in the sagittal plane and likely resulting from obstruction of one or more markers from the cameras' view. Violations of the sphericity assumption were resolved by correcting the degrees of freedom of the statistical test using the Greenhouse-Geisser method.

\subsection{Results}
ANOVA on the frequency-based gait metric derived from the eyeglass camera data revealed no significant differences across gait speed conditions in the vertical plane ($p= .55$, $\eta^2_p = .04$). However, in the frontal plane, there was a significant effect of gait speed, $F(1.49, 22.39) = 36.0$, $p < .001$, $\eta^2_p = .71$.Pairwise post-hoc comparisons revealed significant differences among all three speed conditions (fast vs. medium: Cohen's $d= 1.96$; fast vs. slow: $d= 1.76$; medium vs. slow: $d= 0.75$; all $p< .05$). This indicates that using the frontal plane data ($u$ for the glasses), the sensor was able to distinguish between gait speeds across the three categories. 
 
As was the case with the camera data, the accelerometer data did not discriminate gait speed conditions in the vertical plane ($p= .26$, $\eta^2_p = .09$). There was a significant effect of gait speed in the frontal plane, $F(1.6, 23.98) = 11.43$, $p < .001$, $\eta^2_p = .43$. Post-hoc tests again identified significant differences among all three speed conditions (fast vs. medium: $d = 0.69$; fast vs. slow: $d= 0.99$; medium vs. slow: $d= 0.68$; all $p < .05$).
 
For the motion capture system, we first consider data from the head marker. A significant effect of gait speed condition was observed in the vertical plane, $F(1.41, 21.1) = 160.2$, $p < .001$, $\eta^2_p = .91$. All three speed conditions were found to differ significantly according to post-hoc tests (fast vs. medium: $d= 2.78$; fast vs. slow: $d= 3.51$; medium vs. slow: $d= 2.65$; all $p< .001$). A significant effect was also observed in the frontal plane, $F(1.25, 18.79) = 66.97$, $p < .001$, $\eta^2_p = .82$. All three speed conditions were found to differ significantly according to post-hoc tests(fast vs. medium: $d= 0.78$; fast vs. slow: $d= 3.11$; medium vs. slow: $d= 2.81$; all $p< .01$).
 
For the center of mass displacements calculated from the motion capture data, ANOVA revealed significant effects of gait speed in each plane of motion. In the vertical plane [$F(1.37,20.04) = 175$, $p < .001$, $\eta^2_p = .92$], post-hoc tests revealed significant differences among all conditions (fast vs. medium: $d= 3.17$; fast vs. slow: $d= 3.64$; medium vs. slow: $d= 2.69$; all $p< .001$). Likewise, for the frontal plane [$F(1.19,17.88) = 68.52$, $p< .001$, $\eta^2_p = .82$], all pair-wise post-hoc comparisons were significant (fast vs. medium: $d= 1.06$; fast vs. slow: $d= 2.3$; medium vs. slow: $d= 3.81$; all $p< .05$).

\subsection{Conclusion}
In this particular experiment we evaluated the performance of a gait analysis system which uses only a wearable camera to collect two-dimensional, first-person video. Optical flow and frequency domain analysis were used to generate a dataset from video, and this dataset was then compared to data collected with a wearable tri-axial accelerometer and a 20-camera motion capture system. This is a crucial step to validate the use of a single-camera wearable system against the vest device and the gold standard motion capture system. While both of the latter devices (the accelerometer and motion capture systems) have seen more use in the domain of wearable gait analysis than the wearable camera, the camera-based system has several advantages including cost, simplicity, and the ability to analyze the visual scene to provide context for the gait analysis data. Although the motion-capture system provided superior discrimination of gait speed in all planes of motion, as expected, the eyeglass-based camera system nonetheless discriminated gait speed significantly and outperformed the vest-based accelerometer system. This suggests considerable promise for its use in unobtrusive activity monitoring in a semi-structured environment.

%================================================
\section{Describing Object-centric Activities using Fuzzy Methods}
%================================================
To address our second research question (\emph{Can features extracted from first-person videos recorded in subjects’ homes be used to categorize different interactions with objects?}), we removed the limited scope of a single activity from the previous experiment and brought forward the use of a wearable camera and some of the input features from the first-person video. We focused on identifying the occurrence of activities based on the users' interactions with objects.

Recognizing and understanding human activities occurring in video is an important enabling technology to a variety of computationally intelligent systems. The spectrum of all possible human activities is extremely broad with great variance in both the temporal and spatial aspects of activities. These activities exist on a scale of increasing complexity, with simple (or \textit{atomic}) activities occurring for a short period of time with a limited number of movements, while complex activities may comprise several simultaneous simple activities occurring repeatedly for a longer period of time. Identifying the exact point in time when an atomic activity (e.g. picking up a spoon) turns into a complex activity (e.g. making a cup of coffee) is nearly impossible. This dilemma illustrates an inherent amount of fuzziness in the domain of activity recognition - an activity may be accurately described in multiple ways and at multiple levels of specificity.

In this experiment we sought to address the challenge of handling the natural fuzziness in activity recognition that arises from the overlapping boundaries between activities and their definitions. We use fuzzy logic to discover activities in first-person video of daily activities in the kitchen. Using a variety of extracted features and object annotations from the video, the system is able to describe a diverse set of activities and provide information on the nature of activities without requiring extensive training. We seek to describe activities more completely and reliably than current uninterpretable or black box methods. Such a system can be used to detect activities more effectively in an in-home setting with applications such as aging in place in older adults or other rehabilitation applications.

\subsection{Related Work}
Traditional work in the field of first-person activity recognition is often focused on exploiting information about objects in the scene (i.e. object recognition systems) in order to classify activities being performed \cite{Nakatani2019, McCandless2013, Pirsiavash2012, Sudhakaran2018, Wang2018}. The use of object-based activity models has shown promise, though it suffers from the need to define each classified activity by the objects that are involved. However, visual noise, such as occlusion of the objects, can not only affect the object recognition accuracy, but also hamper the ability of the activity recognition framework as it relies heavily on the identification of objects the human is interacting with. Related works have shown that this noise can be reasonably handled with current object recognition techniques. In fact, \cite{Pirsiavash2012} takes advantage of these changes in appearance to determine whether an interaction is occurring.

Many of the more recent works in this field of study have abandoned traditional video features in favor of deep learning techniques to identify activities occurring in first-person video.   \cite{Li} features a recurrent convolutional neural network (CNN) trained on RGB images, optical flow, and gaze annotations. Similarly, \cite{Ryoo2015} describes the use of convolutional neural network features in a pooled time series representation in combination with a non-linear SVM to identify activities. \cite{Purwanto2017} applies the Hilbert-Huang transform to CNN features and uses an SVM classifier to classify activities. CNN-based techniques are favored for their ability to discover useful features in images and video, but the networks require extensive training. Due to the amount of training required, the set of activities classified by each of these CNN-based systems is fixed and small.

While the deep learning techniques may perform well for a small set of actions when trained against extensive annotated datasets, the requirement of massive amounts of labeled data is a huge limitation in the utility of such methods for activity detection and recognition. Given the multitude of activities that may occur on a daily basis, it is not likely to be feasible to build a training dataset for a system that recognizes a diverse set of these activities. Additionally, when these methods fail to recognize an activity, there is no meaningful output - the deep learning techniques operate as a “black box” and the correlation between features and the output of the system is not well understood outside of that box. Consequently, the identification of an activity has a binary outcome, and when the classification is incorrect, the system provides no useful information.

Fuzzy logic is the means by which we attempt to remove these limitations. In contrast to the pure deep learning approaches, work must be done to extract and understand the video features and their correlation to activities recognized by the system, but the understanding of the parameters provides insight into the type of activity being performed even if the actual name of the activity is unknown. That is, a fuzzy system can provide a result which indicates aspects of the occurring activity without the need to positively identify the activity, incorporating uncertainty that is extremely useful for reliable performance in dynamic environments \cite{Anderson2009}. For many purposes of activity detection, including in-home monitoring of elderly \cite{Banerjee2015, Anderson2009a} and medical patients, human-machine interaction, or life-logging, the fuzzy information may be informative where artificial intelligence techniques produce no useful information.

We take inspiration from existing methods and extract both motion- and object-based features from a set of first-person videos. We describe a fuzzy inference system that uses these features to classify activities occurring in these videos. To our knowledge, the application of fuzzy methods to egocentric activity classification is a novel approach.


\subsection{Methods}
Our goal is to create a fuzzy inference system that classifies a set of activities being performed by the subject of first person videos. We acknowledge that the recognition of an activity is not always a binary decision given the relationships between various atomic and complex activities. For example, multiple atomic activities may occur simultaneously as part of a more complex activity. For this reason, we seek to demonstrate that a fuzzy system of activity recognition which provides a more complete depiction of all aspects of the activities being performed. A simple diagram of our system is provided in Figure \ref{diagram}.

\begin{figure}[t]
\centerline{\includegraphics[width=.9\linewidth]{figure/diagram.png}}
\caption{Illustration of our system. Features such as optical flow motion vectors and color histograms are extracted from video frames and passed to the fuzzy inference system, which calculates fuzzy membership in three outputs based on a series of rules within the system.}
\label{diagram}
\end{figure}

\subsubsection{Features}
We build our fuzzy inference system around a set of features that are extracted from RGB first-person videos. Table \ref{videoFeatures} lists each of the video features which are used as inputs to the fuzzy inference system. Based on the successes of related work, we extract both object features (bounding box, color histogram) and motion features (optical flow). For features that require two video frames as input, such as optical flow or the color histogram difference, we compute the features using a time delta of 15 frames (using frame $t$ and frame $t+15$). This represents a time difference of 0.25 seconds. In our experimentation, a smaller time difference does not typically provide enough motion information, and a larger time difference violates the assumption of a small time gap required for optical flow techniques.

\begin{table}
\caption{Video Features}
\begin{center}
\begin{tabular}{p{3cm}p{5cm}}
\hline \\
\textbf{\textit{Feature}}&\textbf{\textit{Description}} \\
\hline
\hline \\
\textit{Object bounding box} & the location of a rectangular bounding box around each identified object in the video \\ \\
\textit{Object label} & the name of each identified object \\ \\
\textit{Frame optical flow} & the mean, 2-dimensional optical flow vector in the video frame computed between frames $t_0$ and $t_1$\\ \\
\textit{Object optical flow} & the mean, 2-dimensional optical flow vector in the video frame computed between frames $t_0$ and $t_1$ within the bounding box of an object \\ \\
\textit{Color histogram difference} & the distance metric of the color histogram within the bounding box of an object in frames $t_0$ and $t_1$ \\ \\
\textit{Hand intersection} & the ratio of pixels within the bounding box of an object that are identified as belonging to a hand \\
\hline
\end{tabular}
\label{videoFeatures}
\end{center}
\end{table}


The recognition of activities via the fuzzy rules in our system is predicated on the identification of obejcts in the video. The presence and motion of individual objects in the scene are critical inputs to the fuzzy inference system. The fuzzy system is provided with a rectangular bounding box for each object in the scene. The object optical flow feature referenced in Table \ref{videoFeatures}, which is motion occurring within the bounding box area, represents the movement of the object from the persepective of the camera. 

Since the first-person video is recorded by a mobile, body-worn camera, there is expected to be overall motion in the video which is a consequence of the movement of the camera itself. While this motion is not directly used to inform the activity classification process, it must be considered in order to discern the motion of the objects in the scene. This camera motion is approximated by the global optical flow feature in Table \ref{videoFeatures}. Since the camera motion is additive to the motion of the objects, the fuzzy system will consider an object to be moving with the scene only if the perceived motion of the object does not match the global motion of the scene. This implies that the object has some motion relative to its surroundings (presumably through interaction with the object by the subject). Both optical flow features are calculated using the dense Farneback method.

The color histogram difference feature is computed to approximate the amount of change in an object's appearance over a time period. The histogram is computed for all pixels within the object bounding box for two frames, and the difference feature is given by applying the intersection distance metric. This is informative for discerning between activities that involve modifying an object. Previous work has exploited the change in appearance of objects to indicate the occurrence of activities \cite{Pirsiavash2012}.

The object motion and appearance features are used to indicate object interactions, but it is intuitively possible for an object that is not involved in an activity to move or change appearance. To avoid the false detection of activities, a final feature is introduced to approximate the likelihood that each object is being interacted with - the hand intersection feature. Hand regions are segmented from the video based on simple color segmentation. The hand region is intersected with each objects' bounding box, and the intersection feature is computed as the ratio of pixels within the bounding box that coincide with the hand region.

\subsubsection{Fuzzy Membership Functions}
Table \ref{inputFeatures} summarizes the membership functions for the fuzzy input features. The two dimensions of the motion vectors are considered as separate fuzzy variables for each type of motion since vertical and horizontal motion can occur simultaneously. To compensate for potential noise in the optical flow calculations, a ``low'' membership term is used to represent a lack of any significant motion. The fuzzy membership functions for the input features were determined through experimentation and analysis of the input features during sample activities.

\begin{table}
\caption{Input Variables to the Fuzzy Inference System}
\begin{center}
\begin{tabular}{p{2cm}ll}
\hline \\
\textbf{\textit{Input Linguistic Variable}}&\textbf{\textit{Term}}&\textbf{\textit{Membership Function}} \\
\hline
\hline \\
\multirow{3}{2cm}{Frame Motion X (FMX)} & low (LO) & $trap(-0.4, -0.2, 0.2, 0.4)$ \\
 & left (L) & $trap(-5.0, -3.0, -1.0, -0.25)$ \\
 & right (R) & $trap(0.25, 1.0, 3.0, 5.0)$ \\
\hline
\multirow{3}{2cm}{Frame Motion Y (FMY)} & low (LO) & $trap(-0.4, -0.2, 0.2, 0.4)$ \\
 & down (D) & $trap(-5.0, -3.0, -1.0, -0.25)$ \\
 & up (U) & $trap(0.25, 1.0, 3.0, 5.0)$ \\
\hline
\multirow{3}{2cm}{Object Motion X (OMX)} & low (LO) & $trap(-0.4, -0.2, 0.2, 0.4)$ \\
 & left (L) & $trap(-5.0, -3.0, -1.0, -0.25)$ \\
 & right (R) & $trap(0.25, 1.0, 3.0, 5.0)$ \\
\hline
\multirow{3}{2cm}{Object Motion Y (OMY)} & low (LO) & $trap(-0.4, -0.2, 0.2, 0.4)$ \\
 & down (D) & $trap(-5.0, -3.0, -1.0, -0.25)$ \\
 & up (U) & $trap(0.25, 1.0, 3.0, 5.0)$ \\
\hline
\multirow{5}{2cm}{Color Histogram (CH)} & very low (VLO) & $trap(0, 0, 200, 300)$ \\
 & low (LO) & $trap(200, 300, 500, 600)$ \\
 & medium (M) & $trap(500, 600, 900, 1100)$ \\
 & high (HI) & $trap(1000, 1100, 1900, 2000)$ \\
 & very high (VHI) & $trap(1900, 2000, 3000, 3000)$ \\
 \hline
 \multirow{2}{2cm}{Hand Intersection (HINT)} & low (LO) & $tri(0, 0, 0.02)$ \\
 & high (HI) & $tri(0.01, 1, 1)$ \\ \\
\hline
\end{tabular}
\label{inputFeatures}
\end{center}
\end{table}


\subsubsection{Fuzzy Rules}
Our fuzzy inference system contains rules for recognizing activities based on the input features described in table \ref{inputFeatures}, leading to the output of fuzzy membership values for each activity type. It is important to note that since the system is using fuzzy logic, it is possible for a set of inputs to result in high levels of membership in more than one activity type. In this way, the fuzzy system is more descriptive of the activities that are occurring than non-fuzzy counterparts.

The fuzzy rules in our system provide membership levels in the three outputs - object interaction (Table \ref{interactRules}), object modification (Table \ref{modifyRules}), and object relocation (Table \ref{relocateRules}). These outputs each capture distinguishing aspects of various activities. The following examples of interactions with fruit illustrate each category: picking up the fruit is a relocation activity (the object is being moved in the scene); washing the fruit is an interaction activity (the object is being interacted with in a relatively static position within the scene); slicing the fruit is a modification activity (the appearance of the fruit is being physically changed). It is possible that a complex activity may involve a combination or all of these actions, and therefore result in non-zero membership for more than one activity category. This is the benefit of the fuzzy system.

In computing the final outputs of the system, several intermediate consequents are computed. The fuzzy system computes two separate consequents for overall motion of the scene - a vertical and horizontal output, both on the scale [negative, low, positive]. Similar consequents are computed for object motion. These consequents are based on the optical flow input features and indicate whether motion of the scene and object were negative or positive in each direction (up or down, left or right). The ‘low’ membership indicates that the motion was close to zero and is considered stationery.

The fuzzy system includes a rule representing the object’s level of movement within the scene, the output of which is directly tied to the object modification activity category. The frame and object movement consequents are evaluated together to determine membership for this rule. When the magnitude and direction of the frame and object motions are similar, the result is that the object motion is none or low. When the magnitude and/or direction of the frame and object motions differ, the rule defines that the object motion in the scene is medium or high, depending on the level of disagreement.

To utilize the color histogram difference, a simple rule exists in the system which maps the amount of difference in the two histograms to a membership function for the object modification activity category. A large change in the color histogram of an object intuitively indicates a large change in the appearance of the object, and thus the fuzzy system categorizes a high membership in object modification.

\begin{table}
\caption{Rules for the Output ``Interaction'' (\textit{HINT}=hand intersection, \textit{INT}=interaction)}
\begin{center}
\begin{tabular}{cc}
\hline \\
\textbf{\textit{Inputs}}&\textbf{\textit{Outputs}} \\
\hline
\hline \\
\textbf{HINT} & \textbf{INT}\\
\hline \\
 LO & LO \\
 HI & HI \\
\hline
\end{tabular}
\label{interactRules}
\end{center}
\end{table}


\begin{table}
\caption{Rules for the Output ``Modification'' (\textit{INT}=interaction, \textit{CH}=color histogram, \textit{MOD}=modification)}
\begin{center}
\begin{tabular}{ccc}
\hline \\
\multicolumn{2}{c}{\textbf{\textit{Inputs}}}&\textbf{\textit{Outputs}} \\
\hline
\hline \\
\textbf{INT} & \textbf{CH} & \textbf{MOD}\\
\hline \\
LO & & LO \\
 & VLO & LO \\
 & LO & LO \\
 & M & HI \\
 & HI & HI \\
 & VHI & HI \\
\hline
\end{tabular}
\label{modifyRules}
\end{center}
\end{table}


\begin{table}
\caption{Rules for the Output ``Relocation'' (\textit{OMX}=object motion in x direction, \textit{OMY}=object motion in y direction, \textit{FMX}=frame motion in x direction, \textit{FMY}=frame motion in y direction, \textit{INT}=interaction, \textit{REL}=relocation)}
\begin{center}
\begin{tabular}{cccccc}
\hline \\
\multicolumn{5}{c}{\textbf{\textit{Inputs}}}&\textbf{\textit{Outputs}} \\
\hline
\hline \\
\textbf{OMX} & \textbf{OMY} & \textbf{FMX} & \textbf{FMY} & \textbf{INT} & \textbf{REL}\\
\hline \\
 & & & & LO & LO \\
L & & L & & & LO \\
R & & R & & &  LO \\
& U & & U & & LO \\
& D & & D & & LO \\
LO & & LO & & & LO \\
 & LO & & LO & & LO \\

 L & & $\neg$ L & & & HI \\
 R & & $\neg$ R & & & HI \\
 & U & & $\neg$ U & & HI \\
 & D & & $\neg$ D & & HI \\
 LO & & $\neg$LO & & & HI \\
 $\neg$LO & & LO & & & HI \\
 & LO & & $\neg$LO & & HI \\
 & $\neg$LO & & LO & & HI \\
\hline
\end{tabular}
\label{relocateRules}
\end{center}
\end{table}


\subsection{Results}

We evaluated our work against a set of eight videos from the Epic Kitchens dataset \cite{Damen2018ScalingDataset}. This dataset contains unscripted first-person videos of subjects performing daily activities such as cooking, washing dishes, and cleaning in their kitchens. Annotations of objects and actions occurring on the objects are provided every 30 frames. The action annotations provided with the dataset comprise over 100 different verbs. For the purpose of evaluating results against the annotated truth, we binned each annotated truth action into one of three action categories reflecting the fuzzy output variables - interaction, modification, and relocation. While our expectation is that actions will have some membership in more than one output category, for current evaluation purposes we chose the annotated truth for each verb based on the category from which the strongest membership is expected. 

The selected videos contain 4310 action annotations, where each action annotation is defined as a tuple of one of 43 unique verbs and one of 56 unique nouns. Of these actions, 1711 were annotated as `interact', 1225 were annotated as `modify', and 977 were annotated as `relocate'. Table \ref{results} summarizes the results for each class of activity.

From Table \ref{results}, we observe that the fuzzy system produces the highest precision (0.523), recall (0.771), and f1 (0.623) metrics for actions in the `interact' class. However, the specificity for the `interact' class is the lowest (0.537), suggesting that the fuzzy system is identifying a sizeable proportion of false negatives along with the true positives in this class. This indicates that our current system is very sensitive to the true positives with a higher recall but performing poorer due to a higher rate of false positives.

On the flip side, our fuzzy system produced the highest specificity metric for the `relocate' class with a specificity of 0.705. The system is identifying a much lower rate of false positives for this class, but this comes at the expense of identifying true positives; this class also has the lowest precision (0.148), recall (0.175), and f1 (0.160) scores. This indicates that there is a need for further parameter tuning to improve model performance.

Our use of the annotations provided with the dataset also introduces some amount of undesirable noise into the input features. For example, the annotated bounding boxes provided with the dataset are imprecise and only updated every 30 frames; Figure \ref{sampleFrame} illustrates two issues with the bounding box annotations - first, the loose fit of the rectangular bounding box to the objects (Figure \ref{sampleA}), which introduces noise in the object motion input features, and second, the low frequency of updates to the bounding box which produces a poor result when the object moves out of the are before the bounding box is updated (Figure \ref{sampleB}).

The current rules implemented in our system are based on a relatively small amount of information from the video, which limits the ability to properly classify activities in more complex scenes, such as when the image is poorly lit, blurry, or otherwise obscured. A system with more features will be better equipped to tolerate this noise.

While the accuracy of our fuzzy system does not exceed the accuracy reported in related work using supervised classification techniques such as deep learning, the system is competitive with their results and shows potential for use after further refinement \cite{Pirsiavash2012, Koppula2016, Lu2019}. The set of features being supplied to our system is small, and those features are simple yet explainable. A set of more robust features is likely to improve the performance of the system. For example, the hand segmentation feature is computed by a simple color segmentation algorithm. Replacing this with a more robust detection of the hand position will give a more reliable result. Similarly, the use of more complex motion features, per-frame annotations, and non-rectangular bounding boxes are all future work that would improve the classification results.

\begin{table}
\caption{Results of activity classification on first-person video}
\begin{center}
\begin{tabular}{cccccc}
\hline \\
\textbf{\textit{Activity Class}} & \textbf{\textit{precision}} & \textbf{\textit{recall}} & \textbf{\textit{f1}} & \textbf{\textit{specificity}} \\
\hline
interact & 0.523 & 0.771 & 0.623 & 0.537 \\
modify & 0.265 & 0.325 & 0.292 & 0.642 \\
relocate & 0.148 & 0.175 & 0.160 & 0.705 \\
\hline
\end{tabular}
\label{results}
\end{center}
\end{table}

\begin{figure}[t]
\begin{subfigure}{.5\textwidth}
\centerline{\includegraphics[width=.8\linewidth]{figure/frame_4085_cropped.png}}
\caption{Video frame with loose rectangular bounding boxes around a spatula and faucet}
\label{sampleA}
\end{subfigure}
\begin{subfigure}{.5\textwidth}
\centerline{\includegraphics[width=.8\linewidth]{figure/frame_4527_cropped.png}}
\caption{Video frame with object exceeding the rectangular bounding box (the portion of the spatula outside of the bounding box is highlighted in green)}
\label{sampleB}
\end{subfigure}
\caption{Example first-person video frame of subject washing spatula with overlaid bounding boxes, action, and object labels}
\label{sampleFrame}
\end{figure}

During the analysis of our results, we also observed subjectivity in the human-generated action annotations provided with the dataset. The annotation of actions on a per-frame basis is very difficult since there is rarely a definitive start and end to an activity at such close temporal resolution. We observed inconsistencies in the application of action labels stemming from the imprecision of natural language. For example, opening a faucet, opening a refrigerator, and opening a bottle are very different types of actions and motions from the subject. While the verb for each of these actions is the same, they are clearly not the same action. The variance seen between these results via the membership in our fuzzy outputs captures the difference much more precisely than the verb label. Thus, we have identified a possible application of fuzzy methods in the refining of annotated truth.

\subsection{Conclusions}
In this experiment, we took a novel approach to egocentric action classification through the use of a fuzzy inference system, highlighting the potential for the use of the system for in-home monitoring. While the features that were input to the fuzzy system were relatively simple, the system was able to provide fuzzy membership of actions in three unique action categories - object interaction, relocation, and modification - which describe the nature of the interaction. Our evaluation of the system against a set of 4310 unscripted actions revealed the possibility of enhanced action descriptions through fuzzy memberships. This derived understanding can be used for in-home activity analysis in geriatric case, assisted living, or hospital settings. Our evaluation also revealed potential for enhancing the action annotation process. The fuzzy output variables bring the potential to provide a richer description of both atomic and complex activities through varying membership levels in several classes.

Future work with this fuzzy system will involve further parameter tuning using genetic algorithms and enhancing the complexity and number of input features to allow for more complex rules, which will in turn provide a more reliable classification result.



%================================================
%\chapter{Results}
%================================================


%================================================
\chapter{Proposed Future Work}
%================================================
The work described thus far provides introduction and motivation for our problem of describing activities in first-person video, a review of related works in the field including sensor hardware, types of activities, and computational intelligence methods used, an overview of the proposed methods, and the description and results of previous experiments motivated by our research questions. 

While we can see conclusions from completed experiments relating to the first two research questions posed in the Introduction, Research Question 3 (\emph{Can the activity identification system a) accurately describe activities and b) handle uncertainty in the results better than current state-of-the-art methods?}) necessitates further work to improve the accuracy of the fuzzy system described in section 4.2 and to address the question of whether the system handles uncertainty in activity classification results (i.e. whether the system provides enhanced explainability of the results even when the classification is unknown or incorrect).

\section{Timeline}
Remaining activities which are expected to provide a satisfactory response to the outstanding research questions are outlined in Figure \ref{timeline}.

%\begin{landscape}

\begin{figure}
\noindent\resizebox*{\linewidth}{!}{ % Rescale the chart to linewidth
\begin{ganttchart}[vgrid, hgrid, x unit = 1cm]{1}{13}
\gantttitle{2020}{9}
\gantttitle{2021}{4} \\
\gantttitlelist{4,...,12}{1}
\gantttitlelist{1,...,4}{1} \\
\ganttbar{Proposal Defense}{1}{1}\\
\ganttbar{Optimization of Fuzzy Rules using GA}{1}{2} \\
\ganttgroup{Enhancements to Input Features}{2}{7} \\
\ganttbar{Hand recognition with NN}{2}{3} \\
\ganttbar{Advanced Optical Flow Features}{4}{5} \\
\ganttbar{Object Change Features}{6}{7} \\
\ganttbar{Further Rule Discovery and Optimization}{8}{8} \\
\ganttbar{Evaluation of System}{9}{9}\\
\ganttbar{Update Dissertation Document}{10}{11} \\
\ganttbar{Review and Edit Document}{12}{12} \\
\ganttbar{Final Defense}{13}{13} \\
\end{ganttchart}
}
\caption{Proposed project timeline}
\label{timeline}
\end{figure}

A brief description of the events follows:
\begin{itemize}
    \item Optimization of Fuzzy Rules using GA: Refinement and tuning of the existing set of fuzzy rules using genetic algorithms to evolve the membership functions \cite{Tan1999OptimizationRating}. This method optimizes the effectiveness of the existing rules.
    \item Enhancments to Input Features
    \begin{itemize}
        \item Hand recognition with NN: Detecting hands in the video is a strong driver of the classification result. Color segmentation methods were originally chosen for simplicity but are not sufficient across lighting conditions and skin tones. Work has been started on using a neural network to recognize hands with some initial improvement but requires additional tuning to be more effective.
        \item Advanced Optical Flow Features: More complex optical flow-based features have been used in related work \cite{Ryoo2015, Abebe2016}, and these may provide a richer source of data from the observed scene.
        \item Object change features: The use of the color histogram to detect change in an object's appearance may only be reliable for sudden or sharp changes in appearance. Additional object shape descriptors and methods of comparing objects' appearances over time will likely improve the performance of the system.
    \end{itemize}
    \item Further rule discovery and optimization: Following the implementation of new features, a second round of rule optimization and tuning is expected since the new features require new rules.
\end{itemize}

%\end{landscape}

%%=============================
%\chapter{In The Beginning}
%%=============================
%The date of this document generation (current version of this document) is the date of the thesis on page two (\today).{}
%sdafsd sadfsdaf asdfadf. OK, I have nothing to say here, but you should in your thesis/dissertation. Introduce your chapter in a page or so.
%
%On a side note, style files should be located in the texmf tree automatically when a package is installed.  The sty-files used for this template are located in the folder \lq\lq{}sty\_files\_if\_needed\rq\rq{}.  If you do not have the packages already installed, you can also simply move or copy them into the same directory as this file.
%
%%=============================
%\section[Citations/References (short form of section name)]{Citations, using them, referring to them, formatting them, and loving using them because if you don't use them when appropriate it's called \emph{plagiarism}, and references}
%%=============================
%That was a ridiculously long section name to illustrate how to get a shorter version in your table of contents.
%
%See Section~\ref{sec:equations}. I'm not the only one who says this is awesome stuff\cite{Mortara2004}!
%This citation is in the ASME format for which I've included a \verb'bst'
%file. See \textsc{asmems4} in the source document (way near the end). Some  other options are:
%\begin{itemize}
%\item natbib:  put \verb'\usepackage{natbib}' in the header (before \verb'\begin{document}'), and \verb'\bibliographystyle{plainnat}' (just before \verb'\bibliography').  See the natbib documentation for details.
%\item AIAA:\@put \verb'\usepackage{overcite}' in the header (before \verb'\begin{document}'), and \verb'\bibliographystyle{aiaa}' (just before \verb'\bibliography'). Use \verb'\citen{sdfsdf}' for in-line citations. Read the overcite package documentation for more.
%\end{itemize}
%
%A variety of other formats are available on \href{http://www.ctan.org}{CTAN} or though an internet search. You may also just pick one using ``natbib''.\footnote{Don't those quotes look bad before ``natbib''? Well, use two left quotes in \LaTeX\ to get it to look right.}  The formatting of your references is controlled by the \verb'bibliographystyle' command near the end of the document. Use the bibliography style appropriate to your field. On exists, out there. I'm sure. If not (ok, it happened to me 10 years ago), you can use the \verb'makebst' script to make your own. Answer a bunch of questions and it makes a style file for you. If you are using a numerical citation system, you may want to use \verb'\usepackage{cite}'. It creates a condensed numerical list of citations, but can cause conflicts with the \verb'hyperref' package (you may need to decide\ldots sorry, this is a bug that drives me nuts too.)
%
%%=============================
%\subsection{About the Bibliography}
%%=============================
%There are two lines in this section in your \LaTeX\ file. The first is a bibligraphystyle command (see the \textsf{tex} file. Don't move it elsewhere. It won't work out well for you.
%
%You need to choose a style file that formats references the way you want them formatted.  You should chose a style for your major.  Look for bst styles on ctan, or use makebst.sty. This formats the bibliography.
%The last line is the bibliography command. It just tells \LaTeX\  the name of your \textsf{.bib} file. This is a database of your references.
%
%Also, use the \verb'\phantomsection' command to correctly anchor the hyperlink for the bibliography.  Without this, any hyperlinks in the document, and the link from the bookmarks will take you to the incorrect page.
%
%You can have as many sources listed in the *.bib file, but if you do not cite them in your document, they will not show up in your Bibliography.  And if you do, they automatically are sorted\ldots pretty nice!
%
%%=============================
%\subsection{Equations}\label{sec:equations}
%%=============================
%\begin{equation}\label{eq:alphaoverbeta} %You can name the label anything you want, just not the same as something else
%x=\frac{\alpha}{\beta}
%\end{equation}
%Really bad idea: don't start a section with an equation. Don't start a sentence with a variable either. Start with a word. In the body of the text, use the \$ around the variable.  For example, the variable $x$ is the distance. How hard was that?
%
%Sometimes you don't want an equation  number, so use \verb'{equation*}' instead.
%\begin{equation*}
%x=\frac{-b\pm\sqrt{b^2-4ac}}{2a}
%\end{equation*}
%
%If you need to align a set of equations up use the \verb'align' command instead with the use of the \& to set the anchor
%\begin{align}
%x&=\frac{-b\pm\sqrt{b^2-4ac}}{2a}\\
%x&=\frac{\alpha}{\beta}
%\end{align}
%The same applies for the \verb'align' command\ldots if you don\rq{}t want equation numbers,  just use an *.  More can be found at the references listed in Section~\ref{sec:refs}.  You can easily make matrices such as the viscous damping matrix, $C_d$, which is shown in Equation~\eqref{eq:damping_mat}.  Use the \verb'\eqref{eq:\ldots}' command to reference equations properly.
%%
%\begin{equation}
%\label{eq:damping_mat}
%C_{da} =[U_n^Tnn]^{-1}
%\left[ \begin{array}{ccc}
%a&b&c\\
%	\ddots & 				& \\
%	     	& 2 \zeta_i \omega_i 		& \\
%		&				& \ddots \\
%\end{array} \right]^{{-1}} [U_n]^{-1}
%\end{equation}
%
%We don't number equations in \LaTeX. \LaTeX\ does it for us. Label them with names (see the raw \LaTeX\ file).   Just don't put a space in the middle of a variable name.
%
%Now if I have an equation that I want to be between paragraphs, unlike equation (\ref{eq:alphaoverbeta}), I put a blank line after the equation.
%\begin{equation}
%  \label{eq:anothersillyequation}
%  x\neq y
%\end{equation}
%
%See  indent? But if I'm continuing the paragraph, don't put that blank line in
%\begin{equation}
%  \label{eq:anothersillyequation2}
%  x\neq y
%\end{equation}
%and there won\rq{}t be an indent.
%
%If you want a new page and you have a figure you want to stay with the section, you need to use the \verb'\clearpage' command instead of the \verb'\newpage' command.
%
%%=============================
%\section{Figures}
%%=============================
%This is not the same as Section~\ref{sec:equations} on equations. However, if I move that section, I'll still be referring to the right section.
%Better explained by Figure\footnote{a- Don't use footnotes.\ b- Capitalize the word ``Figure''}~\ref{fig:example}.  You can keep all of your figures in a sub-directory such as \lq\lq{}pix\rq\rq{}, which is used in this template.
%\begin{figure}[htbp] %  figure placement: here, top, bottom, or page
%   \centering
%   \includegraphics[width=.5\textwidth]{WSUThesisTemplate/pix/example.pdf} %
%   \caption{Example caption.}
%\label{fig:example}
%\end{figure}
%
%The width of the figure can be set based on percentage of text width as set in Figure~\ref{fig:example} or based on inches as used in Figure~\ref{fig:example2}.  Also notice the \verb'\label' is after the \verb'\caption'.  This must be true, or the hyperlinks and figure numbers will not be correct.
%
%%----------------------------------------------------------
%\begin{figure}[!t] %forced to top of page so no text will be stranded above
%   \centering
%   \includegraphics[width=3in]{WSUThesisTemplate/pix/example.pdf} %
%   \caption[Short Figure Caption]{Example caption that is way too long for the list of figures, is a run on sentence, has no purpose being this long, except to show you how to avoid such a crazy long entry in your list of figures.}
%\label{fig:example2}
%\end{figure}
%%----------------------------------------------------------
%
%Don't ask me why the label command has to come late in a figure. It does.  Remember, color won't print well in black and white. Use dashes and dash-dots, etc, for hard copies. I'll document a trick for this later. Basically, make two graphics directories, one for color, one for black and white. Then, use the \textsf{graphicspath} command to choose the one you want. You can Google this for now.
%
%%=============================
%\section{Sub-figures}
%%=============================
%You can also make sub-figures and reference each of them individually.  You can reference the entire Figure~\ref{fig:1x2_subfigs}, or just Figure~\ref{fig:example_a} or Figure~\ref{fig:example_b}.
%%-----------------------------------------------------------------
%\begin{figure}[!ht]
%  \centering
%  \subfigure[First sub-figure]{
%\label{fig:example_a}
%   \includegraphics[width=.45\textwidth]{WSUThesisTemplate/pix/example.pdf}}
%   \subfigure[Second sub-figure]{
%\label{fig:example_b}
%   \includegraphics[width=.45\textwidth]{WSUThesisTemplate/pix/example.pdf}}
%   \caption{Side-by-side sub-figures.}
%\label{fig:1x2_subfigs}
%\end{figure}
%%-----------------------------------------------------------------
%
%
%If you want 4 total figures, just add a line break, \verb'\\', after the second sub-figure as shown in Figure~\ref{fig:2x2_subfigs}.  You can add spacing between them with the \verb'\quad' or \verb'\qquad' commands.  There is more space between Figures~\ref{fig:example_2x2a} and~\ref{fig:example_2x2b} to show the use of this spacing.  Make sure all of your spacing is equal.  And don't make your figures too small.  As my advisor told me, ``old people read these''\cite{Mark}.
%
%%-----------------------------------------------------------------
%\begin{figure}[!t]
%   \centering
%   \subfigure[First sub-figure]{
%\label{fig:example_2x2a}
%   \includegraphics[width=.25\textwidth]{WSUThesisTemplate/pix/example.pdf}} \quad
%   \subfigure[Second sub-figure]{
%\label{fig:example_2x2b}
%   \includegraphics[width=.25\textwidth]{WSUThesisTemplate/pix/example.pdf}}\\
%      \subfigure[Third sub-figure]{
%\label{fig:example_2x2c}
%   \includegraphics[width=.25\textwidth]{WSUThesisTemplate/pix/example.pdf}}
%   \subfigure[Fourth sub-figure]{
%\label{fig:example_2x2d}
%   \includegraphics[width=.25\textwidth]{WSUThesisTemplate/pix/example.pdf}}
%\caption{2$\times$2 sub-figures.}
%\label{fig:2x2_subfigs}
%\end{figure}
%%-----------------------------------------------------------------
%
%%=============================
%\section{Including Chapters or Files}
%%=============================
%You can include chapters using the \verb'\include' command. See the \LaTeX\ file.  Each file can be included separately as to keep editing localized to each chapter.
%% ------------------------------------------------------------------------
%%\include{chapter2}
%% ------------------------------------------------------------------------
%%\include{chapter3}
%% ------------------------------------------------------------------------
%
%You can also use the \verb'\input' command to include items without forcing a page break.  This becomes handy when generating a table, you can leave the reference in the main document and the table can be updated separately.
%\begin{table}[ht] % h - here, t - top, b - bottom, p - page (use a ! to force the table for figure where you want)
  \centering
  \scriptsize % set font size to scriptsize so that table fits within page margins
  \caption{Complete test matrix of waveforms for experimental bench test}
    \begin{tabular}{cccccc}
    \toprule % top line of table
    \textbf{Bandwidth} & \textbf{FFT lines} & \textbf{Samples/Cycle Chirp} & \textbf{Frequency Resolution} & \textbf{Sweep Rate} & \textbf{Sweep Time} \\
	(Hz) 	& 		&	 		& (mHz) & (mHz/sec)	& (sec) \\ \midrule
\multirow{4}[8]{.5in}{\centering8}
		& 100		& 3-10, 20 		& 80    	& 0.64		& 12.5 \\ \cmidrule (l){2-6} % midrule between columns 2-6
		& 200		& 3-10 	     		& 40
                                                                                                           & 0.32 		& 25 \\ \cmidrule (l){2-6}
		& 400   	& 3-10  		& 20   	& 0.16  	& 50 \\ \midrule
%
\multirow{5}[8]{.5in}{\centering16}
		& 100   	& 3-10, 20, 50      	& 160   & 2.56  	& 6.25 \\ \cmidrule (l){2-6}
		& 200   	& 3-10, 20		& 80    	& 1.28  	& 12.5 \\ \cmidrule (l){2-6}
		& 400   	& 3-10, 20     		& 40    	& 0.64  	& 25 \\ \cmidrule (l){2-6}
		& 800		& 3-10      		 & 20    & 0.32  	& 50 \\ \midrule
%
\multirow{6}[12]{.5in}{\centering32}
		& 100   	& 3-10, 20     		 & 320	& 10.24 	& 3.125 \\ \cmidrule (l){2-6}
		& 200   	& 3-10, 20      		& 160	& 5.12  	& 6.25 \\ \cmidrule (l){2-6}
		& 400   	& 3-10, 20      		& 80	& 2.56  	& 12.5 \\ \cmidrule (l){2-6}
		& 800   	& 3-10, 20      		& 40	& 1.28 	 	& 25 \\ \cmidrule (l){2-6}
		& 1600  	& 3-10       		& 20	& 0.64  	& 50 \\ \bottomrule % bottom line of table
    \end{tabular}
\label{tab:bench_test_matrix}
  \normalsize % reset font size to normal
\end{table}

%
%Using the \verb'booktabs' package makes very professional looking tables by varying the thickness of the lines which can be customized.
%
%%=============================
%\clearpage \section{Inserting Code}
%%=============================
%If you want to insert code into your document by reference, instead of copy/paste, you even use the \href{https://en.wikibooks.org/wiki/LaTeX/Source_Code_Listings}{listings} package.  
%
%\href{http://www.mathworks.com}{Matlab} users may find the simpler interface of the \verb'mcode' package easier.  To do that simply comment the \verb'\usepackage{listings}' line near the top of this (the \verb'.tex') document and uncomment the preceding line. You can choose to between several options to frame, have numbered lines, automatic line breaks and more.  Below is an example of listing a MATLAB\textsuperscript{\textregistered} m-file.
%
%\lstinputlisting[language=matlab, basicstyle=\linespread{1}\normalsize]{WSUThesisTemplate/importfile.m}
%
%%=============================
%\chapter[Programs]{Typesetting Programs using \LaTeX\ }
%%=============================
%\section{Windows}
%%=============================
%Below are some programs for Windows:
%\begin{itemize}
%\item \href{http://miktex.org/}{MiK\TeX}
%\subitem- Up-to-date implementation of \TeX\
%\subitem- Side-by-side comparison of source and PDF
%\subitem- Has portable version that can be run from portable storage device
%\item \href{http://www.lyx.org/}{LyX}
%\subitem- Graphical interface used with \TeX\ and \LaTeX\
%\item \href{http://www.tug.org/texlive/}{\TeX Live} (also Unix)
%\item \href{http://www.tug.org/protext/}{pro\TeX t}
%\end{itemize}
%
%%=============================
%\section{Mac OS}
%%=============================
%Below are some programs for Mac OS:\@
%\begin{itemize}
%\item \href{http://www.lyx.org/}{LyX}
%\item\href{http://www.tug.org/mactex/}{Mac\TeX}
%\subitem- \TeX\ Live with the addition of Mac specific programs
%\item gw\TeX\   (Mac OS X)
%\item Latexian (Mac OS X)
%\end{itemize}
%
%\section{References}
%\label{sec:refs}
%\begin{itemize}
%\item \href{http://www.ctan.org}{CTAN} home page
%\item \href{http://en.wikibooks.org/wiki/LaTeX/}{Wikibooks} \LaTeX\ home page
%\end{itemize}


%
%
%-----------------------------------------------------------------------
% Bibliography
%-----------------------------------------------------------------------
\clearpage \phantomsection\ %used to correctly anchor hyperlinks, just
                           %trust us and leave it alone.
\renewcommand\baselinestretch{1.0}
\addcontentsline{toc}{chapter}{Bibliography}
\bibliographystyle{plain}
\bibliography{references}
%
%-----------------------------------------------------------------------
% Appendices
%-----------------------------------------------------------------------
%%%%%%%%%%%%%%%%%%%
\begin{appendices}
\phantomsection\ %use \phantomsection command to correctly anchor hyperlinks
\include{AppendixA}
\phantomsection\
\include{AppendixB}
\end{appendices}
%%%%%%%%%%%%%%%%%%%
%
%
%
% End of document
\end{document}

%%% Local Variables:
%%% mode: latex
%%% TeX-master: "WSUThesisTemplate"
%%% End:
